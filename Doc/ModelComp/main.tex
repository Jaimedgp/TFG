Los modelos te\'oricos de sistemas f\'isicos permiten entender mejor los procesos observados experimentalmente conociendo m\'as a fondo la f\'isica del problema. Las simulaciones computacionales han supuesto una importante herramienta para el c\'alculo de estos modelos, permitiendo predecir y reproducir los procesos f\'isicos de sistemas complejos. En este cap\'itulo se detallar\'a el modelo te\'orico utilizado para describir la din\'amica del l\'aser de semiconductor \gs. Tambi\'en se realizar\'a una breve explicaci\'on del c\'odigo escrito para la simulación computacional.

	\addtocontents{toc}{\vspace{0.01cm}}
	\section{Modelo de l\'aser de semiconductor encendido por ganancia}
		\label{Mdl:RoF}
	
		La interacción entre los fotones y los portadores en un laser de semiconductor monomodo, incluyendo la fase del campo óptico, puede ser modelado usando la aproximación de las ecuaciones de balance conocida \cite{artSim}, \cite{schunk1986noise}, \cite{fatadin2006numerical}. A continuación se muestran las ecuaciones utilizadas para describir la dinámica de la densidad de portadores ($N(t))$,  la densidad de fotones ($S(t)$) y la fase óptica ($\Phi(t)$).

	\begin{equation}
		\frac{\mathrm{d} N}{\mathrm{d} t} = \frac{I(t)}{e V_{act}} - R(N) - \frac{v_g \textrm{g}(N)S(t)}{1 + \epsilon S(t)} 
		\label{eq:RtEq-N}
	\end{equation}

	\begin{equation}
		\frac{\mathrm{d} S}{\mathrm{d} t} = \left[ \frac{\Gamma v_g \textrm{g}(N)}{1 + \epsilon S(t)} - \frac{1}{\tau_p} \right] S(t) + \beta \Gamma BN^2 (t) + \sqrt{2 \beta \Gamma B N^2(t)S(t)} F_S(t) + Y_S(t)
		\label{eq:RtEq-S}
	\end{equation}

	\begin{equation}
		\frac{\mathrm{d} \Phi}{\mathrm{d} t} = \frac{\alpha}{2}\left[ \Gamma
        v_g \textrm{g}(N) - \frac{1}{\tau_p} \right] + 2\pi\Delta\nu(I) + \sqrt{\frac{\beta \Gamma B N^2(t)}{2 S(t)}} F_{\Phi} (t) + Y_{\Phi}(t)
		\label{eq:RtEq-Ph}
	\end{equation}

	En las ecuaciones \ref{eq:RtEq-N}-\ref{eq:RtEq-Ph} aparecen la corriente inyectada $I(t)$, $V_{act}$ el volumen de la región activa, $e$ la carga del electrón, $R(N)$ velocidad de recombinación de portadores de carga, $v_g$ la velocidad de grupo, $g(N)$ la ganancia del material, $\epsilon$ el coeficiente no lineal de ganancia, $\Gamma$ el factor de confinamiento óptico, $\tau_p$ el tiempo de vida del fotón, $\beta$ la fracción de emisión espontánea acoplada al modo del láser y $\alpha$ el factor de ensanchamiento de línea. Los términos $Y_S(t)$ y $Y_{\Phi}(t)$ describen la inyección óptica externa, y sus expresiones se detallarán más adelante. En la ecuación \ref{eq:RtEq-Ph} se ha incluido un término adicional $\Delta\nu(I)$ que depende de la corriente inyectada  y que es la diferencia de frecuencias entre la frecuencia de emisión dicha corriente y la frecuencia de emisión en la corriente umbral. Para nuestro estudio también resulta interesante la obtención del denominado Chirp en frecuencia, \chirp.

			\begin{equation}
				\nu_{chirp} = \frac{1}{2\pi} \frac{\mathrm{d} \Phi}{\mathrm{d} t}
			\end{equation}%y tiene en cuenta los cambios producidos por la variación de temperatura.

Los términos de Langevin $F_S$ y $F_{\Phi}$ de las ecuaciones \ref{eq:RtEq-S} y \ref{eq:RtEq-Ph}, representan ruidos estocásticos adicionales debidos a la emisión espontánea, con las siguientes relaciones de correlación:

	\begin{equation}
		\left \langle F_i(t)F_j(t') \right \rangle = 2\delta_{ij}\delta(t-t')
		\label{eq:NoiseCorre}
	\end{equation}

En la ecuación \ref{eq:NoiseCorre} $\delta(t)$ es la función delta de Dirac y $\delta_{ij}$ es la función delta de Kronecker, con los subíndices $i$ y $j$ referentes a las variables $S$ y $\Phi$.

La ganancia del material $g(N)$ viene descrita por la ecuación \ref{eq:MatGain}, donde $\frac{\mathrm{d}g}{\mathrm{d}N}$ es la ganancia diferencial y $N_{tr}$ es la densidad de portadores en transpariencia.

	\begin{equation}
		g(N) = \frac{\mathrm{d}g}{\mathrm{d}N} (N - N_{tr})
		\label{eq:MatGain}
	\end{equation}

La recombinación de portadores $R(N)$ se muestra en el ecuación \ref{eq:CarriRcom}, con $A$, $B$ y $C$ los coeficientes de recombinación no radiativa, espontánea y de Auger, respectivamente.

	\begin{equation}
		R(N) = AN + BN^2 + CN^3
		\label{eq:CarriRcom}
	\end{equation}

Si se desprecian los efectos el\'ectricos de alta frecuencia, se puede expresar la corriente inyectada $I(t)$ tal y como se muestra en la ecuación \ref{eq:gainSwtching}.

	\begin{equation}
		I(t) = \ibias + \frac{2\sqrt{2}V_{RF}}{Z_0+Z_l} \sin(2\pi f_R t)
		\label{eq:gainSwtching}
	\end{equation}

	En la ecuación \ref{eq:gainSwtching} $\ibias$ es la corriente en continua aplicada al láser (corriente de polarización), $f_R$ la frecuencia de repetici\'on, $V_{RF}$ es el valor del voltaje (RMS, \textit{root mean squared}) del generador de señal, $Z_0$ la impedancia de salida del generador, y $Z_l$ la impedancia del m\'odulo l\'aser.

	La potencia de salida del l\'aser es obtenida apartir de la densidad de fotones $S(t)$ mediante la ecuaci\'on \ref{eq:Power}, donde $\eta_f$ es la eficiencia cu\'antica externa en fibra dada por el producto de la eficiencia cu\'antica diferencial y la eficiencia de acoplamiento, $h$ es la constante de Planck y $f_0$ es la frecuencia de emisión del l\'aser.

	\begin{equation}
		P(t) = \eta_f \frac{h f_0 V_{act}}{\Gamma \tau_p} S(t)
		\label{eq:Power}
	\end{equation}

	Para los casos en los que se tenga inyecci\'on \'optica, \'esta vendr\'a dada por los t\'erminos $Y_S$ y $Y_{\Phi}$ de las ecuaciones \ref{eq:RtEq-S} y \ref{eq:RtEq-Ph} \cite{schunk1986noise}.

	\begin{equation}
		Y_S = 2k_c\sqrt{S(t)S_{Iny}} \cos(\Phi(t) - 2\pi \delta\nu't)
		\label{eq:Iny-S}
	\end{equation}

	\begin{equation}
		Y_{\Phi} = -k_c\sqrt{\frac{S_{Iny}}{S(t)}} \sin(\Phi(t) - 2\pi \delta\nu't)
		\label{eq:Iny-Phi}
	\end{equation}

En las ecuaciones \ref{eq:Iny-S} y \ref{eq:Iny-Phi} el t\'ermino $S_{Iny}$ hace referencia a la densidad de fotones debida a la inyecci\'on \'optica, que es proporcional a la potencia de inyecci\'on $P_{Iny}$ a partir de la ecuación \ref{eq:Power}. El t\'ermino $k_c$ es el coeficiente de acoplamiento entre el l\'aser maestro y el l\'aser esclavo, y $\delta \nu'$ es la desviaci\'on de la frecuencia del l\'aser maestro respecto a la frecuencia del l\'aser esclavo en el umbral.

De cara a la obtenci\'on de los espectros \'opticos y los peines de frecuencia, se ha descrito el campo \'optico total $E_T(t)$ como una onda con amplitud proporcional a la raíz cuadrada de la potencia de salida $P(t)$ y fase la fase \'optica del l\'aser $\Phi(t)$, tal y como se muestra en la ecuaci\'on \ref{eq:OpField}.

	\begin{equation}
		E_T(t) = \sqrt{P(t)} e^{i \Phi(t)}
		\label{eq:OpField}
	\end{equation}


	\addtocontents{toc}{\vspace{0.01cm}}
	\section{Código de la Simulación}
		\label{Mdl:Code}
	
		El c\'odigo utilizado para la simulaci\'on de la din\'amica del l\'aser de semiconductor \gs\ se ha desarrollado utilizando el lenguaje de programaci\'on Python, versi\'on python 2.7. Los diferentes scripts utilizados para la simulaci\'on \cite{github} permiten resolver las ecuaciones de balance \ref{eq:RtEq-N}-\ref{eq:RtEq-Ph}.

Para el c\'alculo de las ecuaciones de balance se ha utilizado el m\'etodo de resoluci\'on de ecuaciones estoc\'asticas descrito en la secci\'on \ref{Intr:PrcsEstcs}, expandiento las ecuaciones \ref{eq:RtEq-N}-\ref{eq:RtEq-Ph} con las expresiones  \ref{eq:MatGain} \ref{eq:CarriRcom} y \ref{eq:gainSwtching}. Se han incluido tambi\'en los t\'erminos de la inyecci\'on \'optica \ref{eq:Iny-S} y \ref{eq:Iny-Phi}.

	\begin{equation}
		\begin{matrix}
			N(t + \mathrm{d}t) =  & N(t) + \frac{\Delta t \ibias }{e V_{act}} + \frac{\Delta t C_{loss} 2 \sqrt{2} V_{RF}}{eV_{act} (Z_0+Z_l)} \sin(2\pi f_R t) \\ \\
			 & - A\Delta t N(t) - B\Delta t N(t)^2 - C\Delta t N(t)^3 \\ \\
			 & - v_g \frac{\mathrm{d}g}{\mathrm{d}N} \Delta t N(t) \frac{1}{1/S(t) + \epsilon}  + v_g \frac{\mathrm{d}g}{\mathrm{d}N} \Delta t N_{tr} \frac{1}{1/S(t) + \epsilon}
		\end{matrix}
		\label{eq:Code-N}
	\end{equation}

	\begin{equation}
		\begin{matrix}
			S(t + \mathrm{d}t) =  & S(t) + \Gamma v_g \frac{\mathrm{d}g}{\mathrm{d}N} \Delta t N(t) \frac{1}{1/S(t) + \epsilon} - \Gamma v_g \frac{\mathrm{d}g}{\mathrm{d}N} \Delta t N_{tr} \frac{1}{1/S(t) + \epsilon} \\ \\
			 & - \frac{\Delta t}{\tau_p}S(t) + \beta\Gamma B\Delta t N(t)^2 + \sqrt{2 \beta \Gamma B \Delta tN^2(t)S(t)} X_i \\ \\
			 & + 2k_c\sqrt{S(t)S_{Iny}} \cos(\Phi(t) - 2\pi \delta\nu't)
		\end{matrix}
		\label{eq:Code-S}
	\end{equation}

	\begin{equation}
		\begin{matrix}
			\Phi(t + \mathrm{d}t) =  & \Phi(t) + \frac{\alpha}{2}\Gamma v_g \frac{\mathrm{d}g}{\mathrm{d}N} \Delta t N(t) - \frac{\alpha}{2}\Gamma v_g \frac{\mathrm{d} g}{\mathrm{d}N} N_{tr} - \frac{\alpha\Delta t}{2\tau_p} + 2\pi\Delta\nu(I)\Delta t \\ \\
			 & + \sqrt{\frac{\beta \Gamma B \Delta t N^2(t)}{2 S(t))}} Y_i - k_c\sqrt{\frac{S_{Iny}}{S(t)}} \sin(\Phi(t) - 2\pi \delta\nu't)
		\end{matrix}
		\label{eq:Code-Ph}
	\end{equation}

En las ecuaciones \ref{eq:Code-N}-\ref{eq:Code-Ph} aparecen: $\Delta t$ el tiempo de integraci\'on, $\Delta\nu(I)$ la diferencia de frecuencias entre la frecuencia de emisi\'on a $\ibias$ y la frecuencia de emisi\'on en la corriente umbral \cite{Chaves19} y los t\'erminos de ruido gaussiano $X_i$ e $Y_i$ con $N(0, 1)$ e independientes entre s\'i. Para estos t\'erminos de ruido gaussiano $X_i$ e $Y_i$ se ha utlizado la funci\'on \texttt{numpy.random.normal(loc=0, scale=1, size=nTotal)} de la libreria \texttt{NumPy} para Python \cite{numpy}.

Apartir de estas ecuaciones se han obtenido diferentes t\'erminos que no dependen del tiempo (independientes de $N(t)$ y $S(t)$), permitiendo ser calculados antes de la ejecuci\'on de la simulaci\'on. \'Esto se realiza en el script \texttt{Constants.py}, que es importado al realizar la simulaci\'on, ahorrando tiempo de computaci\'on. Con \'este objetivo tambi\'en se han desarrollado las funciones seno y coseno de los t\'erminos de la inyecci\'on teniendo en cuenta las propiedades de esta\'as funciones para resta de \'angulos.

	\begin{equation}
		\begin{matrix}
			\sin(u - v) = \sin(u)\cos(v) - \cus(u)\sin(v) \\ \\

			\cos(u - v) = \cos(u)\cos(v) + \sin(u)\sin(v) 
		\end{matrix}
	\end{equation}

	Los t\'erminos de la inyecci\'on \'optica $Y_S$ e $Y_{\Phi}$ de las ecuaciones \ref{eq:Iny-S} y \ref{eq:Iny-Phi} vienen ccaracterizados por $S_{Iny}$ y $\delta\nu'$. Sin embargo, para un mayor entendimiento en la comparaci\'on con los resultados experimentales, la inyecci\'on \'optica en la simulaci\'on ha sido caracterizada por su potencia inyectada $P_{Iny}$, pudiendo obtener $S_{Iny}$ con la ecuaci\'on \ref{eq:Power}, y por la diferencia de frecuencias $\delta\nu$ entre la frecuencia de inyecci\'on del l\'aser maestro $\nu_{ML}$ y la frecuencia de emisi\'on del l\'aser esclavo sin \gs\ $\nu_{SL}$, que depende de la corriente $\ibias$. Puesto que $\delta\nu'$ viene definido por la frecuencia de emisi\'on del l\'aser esclavo en el umbral $\nu_{th}$, es necesario realizar un cambio de variable.

	\begin{equation}
		\delta\nu' = \delta\nu - \nu_{th} + \nu
	\end{equation}

	Tambi\'en se observa en la ecuaci\'on \ref{eq:Code-N} como la modulaci\'on de la corriente $I(t)$ solo depende del tiempo, por lo que $\sin(2\pi f_R t)$ puede ser calculado y almacenado en un vector al comienzo de la simulaci\'on, ahorrando tiempo de c\'alculo. Se ha considerado $C_{loss} = 1$ para simplificar los c\'alculos \cite{Chaves19}.

	En \texttt{Constants.py} se computan, junto con los t\'erminos nindependientes de $N(t)$ y $S(t)$ de las ecuaciones \ref{eq:Code-N}-\ref{eq:Code-Ph}, el resto de par\'ametros de dichas ecuaciones necesarios para la realizaci\'on de la simulaci\'on, obtenidos de \cite{artSim} y \cite{Chaves19}.

	Puesto que en las ecuaciones \ref{eq:Code-N}-\ref{eq:Code-Ph} se trabaja en referencia a una corriente umbral y una frecuencia umbral, es necesario tener en cuenta los cambios de determinados t\'erminos con la corriente de polarización $\ibias$. En el script \texttt{getDictValues.py} se inicializan los diferentes valores de la frecuencia de emisión $\nu$, la diferencia de frecuencias respecto a la frecuencia de emisi\'on en la corriente umbral \cite{Chaves19} y el corrimiento de frecuencias de la transformada r\'apida de Fourier; en diccionarios de python.

	\addtocontents{toc}{\vspace{0.1cm}}
	\subsection{Transformada R\'apida de Fourier}
		\label{Mdl:Code:Temp}

	\addtocontents{toc}{\vspace{0.1cm}}
	\subsection{T\'ermino de la temperatura}
		\label{Mdl:Code:Temp}

		Explicar el termino de la temperatura

	\addtocontents{toc}{\vspace{0.1cm}}
	\subsection{Transitorio}
		\label{Mdl:Code:Trans}

		Explicar el transitorio

