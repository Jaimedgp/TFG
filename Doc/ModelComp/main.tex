Los modelos te\'oricos de sistemas f\'isicos permiten entender mejor los procesos observados experimentalmente conociendo m\'as a fondo la f\'isica del problema. Las simulaciones computacionales han supuesto una importante herramienta para el c\'alculo de estos modelos, permitiendo predecir y reproducir los procesos f\'isicos de sistemas complejos. En este cap\'itulo se detallar\'a el modelo te\'orico utilizado para describir la din\'amica del l\'aser de semiconductor \gs. Tambi\'en se realizar\'a una breve explicaci\'on del c\'odigo escrito para la simulación computacional.

	\addtocontents{toc}{\vspace{0.01cm}}
	\section{Modelo de l\'aser de semiconductor encendido por ganancia}
		\label{Mdl:RoF}
	
		La interacción entre los fotones y los portadores en un laser de semiconductor monomodo, incluyendo la fase del campo çóptico, puede ser modelado usando la aproximación de las ecuaciones de balance conocida \cite{artSim}, \cite{schunk1986noise}, \cite{fatadin2006numerical}. A continuación se muestran las ecuaciones utilizadas para describir la dinámica de la densidad de portadores ($N(t))$,  la densidad de fotones ($S(t)$) y la fase óptica ($\Phi(t)$).

	\begin{equation}
		\frac{\mathrm{d} N}{\mathrm{d} t} = \frac{I(t)}{e V_{act}} - R(N) - \frac{v_g \textrm{g}(N)S(t)}{1 + \epsilon S(t)} 
		\label{eq:RtEq-N}
	\end{equation}

	\begin{equation}
		\frac{\mathrm{d} S}{\mathrm{d} t} = \left[ \frac{\Gamma v_g \textrm{g}(N)}{1 - \epsilon S(t)} - \frac{1}{\tau_p} \right] S(t) + \beta \Gamma BN^2 (t) + \sqrt{2 \beta \Gamma B N^2(t)S(t)} F_S(t) + Y_S(t)
		\label{eq:RtEq-S}
	\end{equation}

	\begin{equation}
		\frac{\mathrm{d} \Phi}{\mathrm{d} t} = \frac{\alpha}{2}\left[ \Gamma v_g  - \frac{1}{\tau_p} \right] + 2\pi\Delta\nu(I) + \sqrt{\frac{\beta \Gamma B N^2(t)}{2 S(t))}} F_{\Phi} (t) + Y_{\Phi}(t)
		\label{eq:RtEq-Ph}
	\end{equation}

	En las ecuaciones \ref{eq:RtEq-N}-\ref{eq:RtEq-Ph} aparecen la corriente inyectada $I(t)$, $V_{act}$ el volumen activo, $e$ la carga del electrón, $R(N)$ el ratio de recombinación de la carga, $v_g$ la velocidad de grupo, $g(N)$ la ganancia del material, $\epsilon$ el coeficiente no lineal de ganancia, $\Gamma$ el factor de confinamiento óptico, $\tau_p$ el tiempo de vida del fotón, $\beta$ la fracción de emisión espontánea acoplada al modo del láser y $\alpha$ el factor de realzamiento del ancho de línea. Los términos $Y_S(t)$ y $Y_{\Phi}(t)$ describen la inyección óptica externa, y sus expresiones se detallarán más adelante. En la ecuación \ref{eq:RtEq-Ph} se ha incluido un término adicional $C(I)$ que depende de la corriente inyectada y tiene en cuenta los cambios producidos por la variación de temperatura.

Los términos de Langevin $F_S$ y $F_{\Phi}$ de las ecuaciones \ref{eq:RtEq-S} y \ref{eq:RtEq-Ph}, representan ruidos estocásticos adicionales provenientes de diferentes fuentes, con las siguientes relaciones de correlación:

	\begin{equation}
		\left \langle F_i(t)F_j(t') \right \rangle = 2\delta_{ij}\delta(t'-t)
		\label{eq:NoiseCorre}
	\end{equation}

En la ecuación \ref{eq:NoiseCorre} $\delta(t)$ es la función delta de Dirac y $\delta_{ij}$ es la función delta de Kronecker, con los subíndices $i$ y $j$ referentes a las variables $S$ y $\Phi$.

La ganancia del material $g(N)$ viene descrita por la ecuación \ref{eq:MatGain}, donde $\frac{\mathrm{d}g}{\mathrm{d}N}$ es la ganancia diferencial y $N_{tr}$ es la densidad de portadores en transpariencia.

	\begin{equation}
		g(N) = \frac{\mathrm{d}g}{\mathrm{d}N} (N - N_{tr})
		\label{eq:MatGain}
	\end{equation}

La recombinación de portadores $R(N)$ se muestra en el ecuación \ref{eq:CarriRcom}, con $A$, $B$ y $C$ los coeficientes de recombinación no radiativa, espontánea y de Auger, respectivamente.

	\begin{equation}
		R(N) = AN + BN^2 + CN^3
		\label{eq:CarriRcom}
	\end{equation}

Si se desprecian los efectos el\'ectricos de alta frecuencia, se puede espresar la frecuencia inyectada $I(t)$ tal y como se muestra en la ecuación \ref{eq:gainSwtching}.

	\begin{equation}
		I(t) = \ibias + \frac{2\sqrt{2}V_{RF}}{Z_0+Z_l} \sin(2\pi f_R t)
		\label{eq:gainSwtching}
	\end{equation}

	En la ecuación \ref{eq:gainSwtching} $\ibias$ es la corriente de polarización, $f_R$ la frecuencia de repetici\'on, $C_{loss}$ un coeficiente de p\'erdida que tiene en cuenta la dependencia con la frecuencia de la atenuaci\'on el\'ectrica de los cables, $V_{RF}$ la media cuadr\'atica del valor del voltaje del generador de señal aplicado a una carga ideal de $50 \Omega$, $Z_0$ la impedancia de salida del generador, y $Z_l$ la impedancia del m\'odulo l\'aser.

	La potencia de salida del l\'aser es obtenida apartir de la densidad de fotones $S(t)$ mediante la ecuaci\'on \ref{eq:Power}, donde $\eta_f$ es la eficiencia cu\'antica externa en fibra dada por el producto de la eficiencia cu\'antica diferencial y la eficiencia de acoplamineto, $h$ es la constante de Plank y $f_0$ es la frecuencia de emisión del l\'aser.

	\begin{equation}
		P(t) = \eta_f \frac{h f_0 V_{act}}{\Gamma \tau_p} S(t)
		\label{eq:Power}
	\end{equation}

	Para los casos en los que se tenga inyecci\'on \'optica, \'esta vendr\'a dada por los t\'erminos $Y_S$ y $Y_{\Phi}$ de las ecuaciones \ref{eq:RtEq-S} y \ref{eq:RtEq-S} \cite{schunk1986noise}.

	\begin{equation}
		Y_S = 2k_c\sqrt{S(t)S_{Iny}} \cos(\Phi(t) - 2\pi \delta\nu't)
		\label{eq:Iny-S}
	\end{equation}

	\begin{equation}
		Y_{\Phi} = -k_c\sqrt{\frac{S_{Iny}}{S(t)}} \sin(\Phi(t) - 2\pi \delta\nu't)
		\label{eq:Iny-Phi}
	\end{equation}

En las ecuaciones \ref{eq:Iny-S} y \ref{eq:Iny-Phi} el t\'ermino $S_{Iny}$ hace referencia a la densidad de fotones debida a la inyecci\'on \'optica, que es proporcional a la potencia de inyecci\'on $P_{Iny}$ a partir de la ecuación \ref{eq:Power}. El t\'ermino $k_c$ es el coeficiente de ecoplamiento entre el l\'aser maestro y el l\'aser esclavo, y $\delta \nu'$ es la desviaci\'on del campo del l\'aser inyectado respecto a la frecuencia del l\'aser esclavo en el umbral.

De cara a la obtenci\'on de los espectros \'opticos y los peines de frecuencia, se ha descrito el campo \'optico total $E_T(t)$ como una onda con amplitud proporcional a la potencia de salida $P(t)$ y fase la fase \'optica del l\'aser $\Phi(t)$, tal y como se muestra en la ecuaci\'on \ref{eq:OpField}.

	\begin{equation}
		E_T(t) = \sqrt{P(t)} e^{i \Phi(t)}
		\label{eq:OpField}
	\end{equation}


	\addtocontents{toc}{\vspace{0.01cm}}
	\section{Código de la Simulación}
		\label{Mdl:Code}
	
		\definecolor{gray97}{gray}{.97}
\definecolor{gray75}{gray}{.75}
\definecolor{gray45}{gray}{.45}

\lstset{ frame=Ltb,
	framerule=0pt,
	aboveskip=0.1cm,
	framextopmargin=3pt,
	framexbottommargin=3pt,
	framexleftmargin=0.4cm,
	framesep=0pt,
	rulesep=.4pt,
	backgroundcolor=\color{gray97},
	rulesepcolor=\color{black},
	%
	stringstyle=\ttfamily,
	showstringspaces = false,
	basicstyle=\small\ttfamily,
	commentstyle=\color{gray45},
	keywordstyle=\bfseries,
	%
	numbers=left,
	numbersep=15pt,
	numberstyle=\tiny,
	numberfirstline = false,
	breaklines=true,
}

\lstdefinestyle{Python}
{basicstyle=\scriptsize\bf\ttfamily,
language=python,
}

\centering\small
\vspace{0.4cm} +- - - - - - - - - - - - - 18 l\'ineas - - - - - - - - - - - - -+
\lstinputlisting[style=Python, firstline=19, lastline=26, firstnumber=19]{../src/simulation.py}
\vspace{-0.4cm} +- - - - - - - - - - - - - 18 l\'ineas - - - - - - - - - - - - -+
\lstinputlisting[style=Python, firstline=44, lastline=53, firstnumber=44]{../src/simulation.py}

\vspace{-0.4cm} +- - - - - - - - - - - - - 24 l\'ineas - - - - - - - - - - - - -+
\lstinputlisting[style=Python, firstline=77, lastline=81, firstnumber=77]{../src/simulation.py}

\vspace{-0.4cm} +- - - - - - - - - - - - - 15 l\'ineas - - - - - - - - - - - - -+
\lstinputlisting[style=Python, firstline=96, lastline=96, firstnumber=96]{../src/simulation.py}

\vspace{-0.4cm} +- - - - - - - - - - - - - 20 l\'ineas - - - - - - - - - - - - -+
\newpage
\lstinputlisting[style=Python, firstline=116, lastline=155, firstnumber=116]{../src/simulation.py}

\vspace{-0.4cm} +- - - - - - - - - - - - - 10 l\'ineas - - - - - - - - - - - - -+
\newpage
\lstinputlisting[style=Python, firstline=165, lastline=192, firstnumber=165]{../src/simulation.py}

\vspace{-0.4cm} +- - - - - - - - - - - - - 10 l\'ineas - - - - - - - - - - - - -+
\lstinputlisting[style=Python, firstline=202, lastline=206, firstnumber=202]{../src/simulation.py}

\vspace{-0.4cm} +- - - - - - - - - - - - -  6 l\'ineas - - - - - - - - - - - - -+
\lstinputlisting[style=Python, firstline=212, lastline=220, firstnumber=212]{../src/simulation.py}

\vspace{-0.4cm} +- - - - - - - - - - - - - 39 l\'ineas - - - - - - - - - - - - -+
%\lstinputlisting[style=Python, firstline=96, lastline=220, firstnumber=100]{../src/simulation.py}

