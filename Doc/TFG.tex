%----------------------------------------------------------------------------------------
%	PACKAGES AND OTHER DOCUMENT CONFIGURATIONS
%----------------------------------------------------------------------------------------

	\documentclass[a4paper, 11pt, oneside]{book}

	\usepackage[sc]{mathpazo} % Use the Palatino font
	\usepackage[utf8]{inputenc}
	\usepackage[spanish, es-tabla]{babel}
	\decimalpoint
	
	\usepackage{lipsum}
	\usepackage{graphicx}
	\usepackage[T1]{fontenc} % Use 8-bit encoding that has 256 glyphs
	\linespread{1.15} % Line spacing - Palatino needs more space between lines
	%\usepackage{microtype} % Slightly tweak font spacing for aesthetics
	%
	\usepackage[hmarginratio=1:1,left=20mm, top=22mm]{geometry} % Document margins
	%\usepackage{multicol} % Used for the two-column layout of the document
	\usepackage[hang, small,labelfont=bf,up,textfont=it,up]{caption} % Custom captions under/above floats in tables or figures
	%\usepackage{mathtools}
	%\usepackage{booktabs} % Horizontal rules in tables
	\usepackage{float} % Required for tables and figures in the multi-column environment - they need to be placed in specific locations with the [H] (e.g. \begin{table}[H])
	\usepackage{hyperref} % For hyperlinks in the PDF
	\usepackage{appendix}
	\usepackage{subcaption}
	%\usepackage{wrapfig}
	%%\usepackage[]{mcode} % For embebing matlab code
	%\usepackage[makeroom]{cancel}
	%
	%%\usepackage{lettrine} % The lettrine is the first enlarged letter at the beginning of the text
	%\usepackage{paralist} % Used for the compactitem environment which makes bullet points with less space between them
	%
	%\usepackage{abstract} % Allows abstract customization
	%\renewcommand{\abstractnamefont}{\normalfont\bfseries} % Set the "Abstract" text to bold
	%\renewcommand{\abstracttextfont}{\normalfont\small\itshape} % Set the abstract itself to small italic text
	%%%
	%\usepackage{titlesec} % Allows customization of titles
	%%\renewcommand\thesection{\Roman{section}} % Roman numerals for the sections
	%%\renewcommand\thesubsection{\Roman{subsection}} % Roman numerals for subsections
	%\titleformat{\section}[block]{\large\scshape\centering}{\thesection.}{1em}{} % Change the look of the section titles
	%\titleformat{\subsection}[block]{\large\centering}{\thesubsection.}{1em}{} % Change the look of the section titles
	%
	%\usepackage{fancyhdr} % Headers and footers
	%\pagestyle{fancy} % All pages have headers and footers
	%\fancyhead{} % Blank out the default header
	%\fancyfoot{} % Blank out the default footer
	%\fancyhead[C]{Titulo Corto% based on TRACS 
	%\hspace{4pt} $\bullet$ \hspace{4pt} MES ANO } % Custom header text
	%\fancyfoot[RO,LE]{\thepage} % Custom footer text
	%
	\usepackage{cite}
	\usepackage{listings}
	\usepackage{color}

	\graphicspath{{../Graphics/}}
	%
	%\DeclareGraphicsExtensions{.pdf,.png,.jpg} % Graphics type

%----------------------------------------------------------------------------------------
%	   METADATA
%----------------------------------------------------------------------------------------

	%\title{
	%	\selectfont\textbf{TITULO}% Article title
	%}
%
	%\author{
	%	\textsc{Jaime Diez Gonzalez-Pardo}\\
	%	%\thanks{A thank you or further information}\\ % Your name
	%	\fontsize{28pt}{10pt} Universidad de Cantabria \\ % Your institution
	%	%\thanks{A thank you or further information}\\[2mm] % Your name
	%	\normalsize gsdfgdfgdfgatura \\ 
	%	\normalsize{Compañeros:} \textsc{NOMBRE COMPANEROS }\\
	%	%\vspace{5mm}
	%}
%
	%\date{\today}

%----------------------------------------------------------------------------------------
%      · DOCUMENT
%----------------------------------------------------------------------------------------

	\begin{document}

		\begin{titlepage} 

			\newcommand{\HRule}{\rule{\linewidth}{0.5mm}} 
			
			\center % Centre everything on the page
			
			%------------------------------------------------
			%	Headings
			%------------------------------------------------
			
				%------------------------------------------------
				%	Logo
				%------------------------------------------------
				
					\begin{figure}[H]
						\centering
						\includegraphics[scale=0.6]{download.png}
					\end{figure}

					\textsc{\LARGE Facultad de Ciencias}\\[1.5cm] 
			
			
			%------------------------------------------------
			%	Title
			%------------------------------------------------
			
				\HRule\\[0.4cm]
				
				{\huge\bfseries Simulación de peines de frecuencia óptica generados por láseres de semiconductor}\\[0.8cm] % Title of your document

				{\huge Simulation of optical frequecy comb generate by semiconductor lasers}\\[0.4cm] % Title of your document
				
				\HRule\\[1.5cm]

				{\Large Trabajo de Fin de Grado para acceder al}\\[0.4cm]

				{\LARGE\bfseries Grado en Física}\\[3cm]
			
			%------------------------------------------------
			%	Author(s)
			%------------------------------------------------
			
				\begin{flushright}
					\large
					\textit{Autor}\\
					Jaime \textsc{DÍEZ GONZÁLEZ-PARDO} \\ % Your name
					\large
					\textit{Director}\\
					Dr. Á.A. \textsc{VALLE GUTIERREZ} % Supervisor's name
				\end{flushright}
			
			%------------------------------------------------
			%	Date
			%------------------------------------------------
			
				\vfill\vfill\vfill % Position the date 3/4 down the remaining page
				
				{\large\today} % Date, change the \today to a set date if you want to be precise
				
			%----------------------------------------------------------------------------------------
			
				\vfill % Push the date up 1/4 of the remaining page
		
		\end{titlepage}

		%\thispagestyle{fancy} % All pages have headers and footers

		\tableofcontents

		\listoffigures

		%----------------------------------------------------------------------------------------
		%	  ARTICLE CONTENTS
		%----------------------------------------------------------------------------------------

			\addtocontents{toc}{\vspace{0.1cm}}
			\chapter{Introducción} % Scope of the project = rad effects + minimization
				\label{sec:Intro}

				%\input{}

				\addtocontents{toc}{\vspace{0.1cm}}
				\section{Láseres de Semiconductor}
					
					%\input{}

				\addtocontents{toc}{\vspace{0.1cm}}
				\section{Procesos Estocásticos}
					
					%\input{}

				\addtocontents{toc}{\vspace{0.1cm}}
				\section{Dinámica No Lineal}
					
					%\input{}
%
				\addtocontents{toc}{\vspace{0.1cm}}
				\section{Peines de Frecuencia Óptica}
					
					

Las l\'ineas de emisi\'on de un l\'aser de semiconductor se obtienen del espectro \'optico, definido como el m\'odulo cuadrado de la transformada de Fourier del campo el\'ectrico, que depende de la potencia del l\'aser y de la fase \'optica de \'este. Para un l\'aser en corriente cont\'inua, el campo el\'ectrico, sin ningún tipo de ruido,  alcanza un valor constante y as\'i, su transformada de Fourier es una funci\'on delta de Dirac $\delta(\nu-\nu_0)$, obteniendo en el espectro una \'unica l\'inea de emisi\'on bien definida para una cierta frecuencia $\nu_0$.

	\begin{figure}[H]
		\centering
		\includegraphics[width=1.0\linewidth]{OFC.png}
		\caption{\label{Img:FFtPulsos}Esquemas de la relaci\'on entre la funci\'on de la evolución temporal de una señal (fila superior) y su transformada de Foutier en el espacio de frecuencias $\nu$ (fila inferior), para tres tipos de señales: (a) tren de pulsos infinitamente estrechos, (b) un \'unico pulso con anchura (FWHM) $\delta t$ y (c) un tren de pulsos con anchura (FWHM) $\delta t$.}
	\end{figure}

Si se tiene una señal formada por un tren de pulsos infinitamente estrechos separados un tiempo $T$, se sabe por su transformada de Fourier que el espectro \'optico en frecuencias ha de ser otro tren de pulsos infinitamente estrechos, pero separados $1/T$ (Figura \ref{Img:FFtPulsos} (a)). Sin embargo, en los experimentos no es posible obtener pulsos infinitamente estrechos sino que se tiene pulsos con una cierta anchura FWHM (siglas en ingl\'es de anchura a media altura). Su espectro \'optico ser\'a otro pulso con FWHM $\delta\nu$ (Figura \ref{Img:FFtPulsos} (b)), donde se tiene que se ha de cumplir que el producto $\delta\nu\delta t$ es contante. En la ecuaci\'on \ref{eq:types} se muestran diferentes valores del producto $\delta\nu\delta t$ seg\'un la forma del pulso en $t$.

	\begin{equation}
		\delta\nu\delta t = 
			\left\{\begin{matrix}
				& 0.315 & \textrm{sech}^2 \\
				& 0.44 & \textrm{Gaussiana}
			\end{matrix}\right.
		\label{eq:types}
	\end{equation}

	Un tren de pulsos con ancho $\delta t$ ser\'a la combinaci\'on de las dos señales anteriores y su expresi\'on vendr\'a dada por la convoluci\'on de ambas funciones. Su transformada de Fourier es el producto de las transformadas de Fourier de las funciones anteriores, obteniendo un espectro formado por pulsos estrechos con una separaci\'on entre ellos de $1/T$ y cuya envolvente viene dada por la forma del espectro del pulso individual (Figura \ref{Img:FFtPulsos} (c)). A este tipo de espectros se les denomina peines.

Los peines de frecuencia \'optica (OFC por sus siglas en ingl\'es) son fuentes \'opticas formados por un gran n\'umero l\'ineas de emisi\'on con un espaciado preciso y equidistante. Los OFC vienen caracterizados tanto por la forma de la envolvente como por la separaci\'on entre los picos, obteniendo OFC de mayor calidad para envolventes anchas y regulares. Esto se obtiene para campos el\'ectricos con pulsos regulares, estrechos e intensos.

Cabe recordar que los espectros \'opticos se obtienen a partir del campo el\'ectrico, cuyo comportamiento puede ser descrito mediante una ecuaci\'on de ondas de la que obtenemos una fase \'optica $\Phi$ y una amplitud relacionada con la potencia del l\'aser. Mientras que para la potencia puede ser descrita con el desarrollo anterior, los efectos de la fase \'optica son diferentes. De esta forma, los efectos que se observan al estudiar los OFC son el resultado de la evolución de amplitud y fase óptica. Uno de los efectos de la fase \'optica en los OFC se puede observar en la anchura de las l\'ineas espectrales, que disminuye para fases menos aleatorias.

Pese a que existen diferentes mecanismos de creaci\'on de OFC, en este trabajo se estudiar\'an solo los m\'etodos de generaci\'on de OFC mediante \gs\ y por inyecci\'on \'optica.

	\subsection{Encendido por Ganancia}
		\label{Intr:OFC:GS}

		El \gs\ (\textit{Gain-Switching} en ingl\'es) es una t\'ecnica mediante la cu\'al se alcanza r\'apidamente un alto valor para la ganancia del l\'aser \cite{principles}. Esta t\'ecnica permite generar pulsos del l\'aser de corta duraci\'on y grandes picos de potencia, pudiendo obtener OFC de gran calidad. El \gs\ consiste en conseguir que la inversi\'on de poblaci\'on, y por tanto la ganancia, alcance un valor muy por encima del valor umbral antes de que la densidad de fotones tenga tiempo de alcanzar un nivel suficiente para reducir la inversi\'on. 
		
		El uso de pulsos de bombeo suficientemente r\'apidos, corriente eléctrica en el caso de láseres de semiconductor, permite alcanzar la condici\'on de la inversi\'on de poblaci\'on, alcanzando el \gs. Mediante el uso de corriente de inyecci\'on modulada por una funci\'on sinusoidal se puede controlar la forma del pulso de bombeo a partir de la amplitud y frecuencia, controlando los valores \'optimos de ancho y potencia de pulsos ópticos para el \gs.

	\subsection{Inyección Óptica}

		Otro m\'etodo de generaci\'on de OFC es mediante la inyecci\'on \'optica. \'Esta consiste en inyectar fotones provenientes de un segundo l\'aser al l\'aser de semiconductor. Bajo determinadas condiciones de potencia y frecuencia del láser que inyecta, se pueden obtener OFC. Puesto que se va a trabajar con l\'aseres de semiconductor, cabe destacar la gran sensibilidad de \'estos a la inyecci\'on \'optica debida entre otras cosas al acoplamiento amplitud-fase a trav\'es del factor de ensanchamiento del ancho de l\'inea \cite{tfgPopp}.

		Para un l\'aser sin inyecci\'on de luz, la fase evoluciona de manera aleatoria en el tiempo, principalmente debido a la emisión espontánea, y as\'i se obtienen anchos de l\'inea grandes (del orden del MHz para un láser monomodo). Sin embargo, al realizarse la inyecci\'on de luz las caracter\'isticas de la fase del l\'aser inyectado pasan a estar determinadas por la inyecci\'on , pudiendo obtener un fase menos aleatoria si el espectro óptico del láser que inyecta es más estrecho. De esta forma se tiene un menor ruido en la fase, obteniendo picos m\'as estrechos en el OFC.

		Este m\'etodo tambi\'en puede producir otro fen\'omeno bajo unas condiciones determinadas, conocido como bloqueo por inyecci\'on. Si se tiene una inyecci\'on \'optica de fotones de frecuencia diferente a la del l\'aser inyectado, bajo ciertas condiciones se puede dar que el l\'aser inyectado comience a emitir en la frecuencia del l\'aser que inyecta, desapareciendo la emisi\'on del l\'aser inyectado a la frecuencia de emisión en solitario.

	\subsection{Aplicaciones}

		Tal y como se ha descrito anteriormente, los OFC presentan l\'ineas de emisi\'on bien definidas, perfectamente equiespaciadas y con una fuerte correlaci\'on en la fase \cite{desi2017development}. \'Esto les convierte en una herramienta de gran inter\'es para la espectroscop\'ia y las cominucaciones \'opticas.

		Los OFC permiten obtener varias l\'ineas de emisi\'on bien definidas y equiespaciadas para diferentes longitudes de onda a partir de la emisi\'on de un \'unico l\'aser. De esta forma permite sustituir sistemas formados por multiples l\'aseres independientes, disminuyendo costes, consumo de potencia y complejidad. Estos sistemas con láseres independientes se utilizan en la actualidad en comunicaciones ópticas de alta velocidad y se les llama DWDM (\textit{Dnse wavelength division multiplexing} en inglés). Si a esto se le añade la capacidad de ajustar la frecuencia de separación entre las líneas o la longitud de onda de éstas, se obtiene unas cualidades de gran importancia para su uso en comunicaciones \'opticas. Esta capacidad de ajuste se puede obtener con los OFC en láseres de semiconductor en \gs. Además, la correlación de fases del OFC presenta multiples ventajas para la transmisión por fibra óptica, pudiendo ser utilizada para cancelar los efectos no lineales que distorsionan la información que viaja por la fibra, pudiendo recuperar dicha información \cite{temprana2015overcoming}. Para las aplicaciones en comunicaciones ópticas de alta velocidad, son deseables OFC con una gran separación entre líneas. Sin embargo, para las aplicaciones de espectroscopía óptica se requieren OFC con separación entre líneas pequeña, para obtener espectros con buena resolución.


				\addtocontents{toc}{\vspace{0.1cm}}
				\section{Objetivo del Estudio}
					
					%\input{}
				
			\addtocontents{toc}{\vspace{0.1cm}}
			\chapter{Modelo Computacional}

				\addtocontents{toc}{\vspace{0.1cm}}
				\section{RoF}
				
					%\input{}

				\addtocontents{toc}{\vspace{0.1cm}}
				\section{Código de la Simulación}
				
					%\input{}
				
			\addtocontents{toc}{\vspace{0.1cm}}
			\chapter{Láser en solitario}

				
Para el estudio del m\'etodo de generaci\'on de OFC mediante \gs\ se ha trabajado con una corriente de inyecci\'on $I(t)$ modulada mediante una función sinusoidal superpuesta a una corriente de polarización $\ibias$ tal y como se muestra en la ecuación \ref{eq:gainSwtching}.

	\begin{equation}
		I(t) = \ibias + \frac{2\sqrt{2}V_{RF}}{Z_0+Z_l} \sin(2\pi f_R t)
		\label{eq:gainSwtching}
	\end{equation}

	Tal y como se vi\'o en el apartado \ref{Intr:OFC:GS} la calidad del \gs\ viene dada tanto por la intensidad de los picos como por la duraci\'on del pulso. De esta manera, se ha procedido a caracterizar los peines \'opticos de frecuencia en funci\'on del \gs\ aplicado modificando la frecuencia de oscilaci\'on y la amplitud de la corriente inyectada. Para el estudio del \gs\ en función de la frecuencia de oscilaci\'on se ha modificado el valor de $f_R$, estudiando primero los OFC para altas frecuencias ($f_R = 5.0$ GHz) y luego para bajas frecuencias ($f_R = 500$ MHz). Cabe destacar que al variar el valor de la frecuencia de oscilaci\'on $f_R$, la impedancia del l\'aser $Z_l$ tambi\'en cambia y as\'i también la suma $Z_0 + Z_l$.

	Para ambos valores de frecuencias $f_R$ se han estudiado los efectos producidos al variar la amplitud de la corriente de inyecci\'on, comparando tanto los espectros ópticos obtenidos como las variables dinámicas para diferentes amplitudes. Para el estudio con diferentes amplitudes ha bastado con modificar los valores de $V_{RF}$, ya que $(Z_0 + Z_l)$ solo varia para la frecuencia.

	\addtocontents{toc}{\vspace{0.1cm}}
	\subsection{Efecto de la amplitud de modulación a altas frecuencias}
		\label{Sol:OFC:HgFreq}

		Para el estudio del efecto de la amplitud de modulación a altas frecuencias se ha trabajado con una corriente de polarización $\ibias = 30$ mA y una frecuencia $f_R = 5.0$ GHz. Tal y como se vi\'o en el apartado \ref{Sol:CW:RoF}, la frecuencia de oscilaciones de relajación del l\'aser para $\ibias = 30$ mA es de $\nu_{RoF} \approx 5.9$ GHz, del orden de $f_R$. Se han resulto las ecuaciones de balance, obteniendo los OFC para tres amplitudes diferentes con $V_{RF}$: $0.05$ V, $1.00$ V y $1.50$ V. 

		En la Figura \ref{Img:rateEquations} se muestra la evolución temporal de la \I, la \s, la \n\ y de \chirp\ para varios valores de $V_{RF}$ pasada la zona del transitorio.

			% Img:rateEquations
			\begin{figure}[H]
				\centering
				\includegraphics[width=1.0\linewidth]{rateEquations.png}
				\caption{\label{Img:rateEquations}Evolución temporal de la \I ((a)-(c)), la \s ((d)-(f)), la \n\ ((g)-(i)) y del \chirp\ ((j)-(l)) en funci\'on de $V_{RF}$ pasada la zona del transitorio. Para la \I\ se ha marcado la corriente umbral del l\'aser $I_{th} = 14.8$ mA con una l\'inea horizontal discontinua. En la primera columna se muestran las evoluciones temporales para una amplitud de la corriente equivalente a $V_{RF} = 0.05$ V (verde), en la segunda columna para $V_{RF} = 1.00$ V (azul) y en la tercera columna de $V_{RF} = 1.50$ V(naraja).}	
			\end{figure}

		Mientras que para el caso del l\'aser en corriente continua ($I(t) = \ibias$) estudiado en la secci\'on anterior (secci\'on \ref{Sol:CW}), $S(t)$, $N(t)$ y el \chirp\ alcanzaban un valor constante pasado el transitorio, ahora la modulación en la corriente produce oscilaciones de igual periodo en $S(t)$, $N(t)$ y el \chirp. Se observa un aumento de la amplitud en $S(t)$, $N(t)$ y el \chirp\ al aumentar la amplitud de la corriente. Adem\'as, se observa como las oscilaciones en $N(t)$ y el \chirp\ van en fase (m\'aximos en el mismo tiempo $t$), mientras que los m\'aximos de $S(t)$ se obtienen cuando $N(t)$ decae a $N_{th}$.
			
		Para el caso de $V_{RF} = 0.05$ V, con una menor amplitud, se observa que las oscilaciones en la corriente (Figura \ref{Img:rateEquations} (a)) son pequeñas. Al igual que la corriente; la \s, la \n\ y el \chirp\ tambi\'en presentan oscilaciones de amplitud pequeña.% Cabe destacar que para este caso de amplitud pequeña, la \s\ (Figura \ref{Img:rateEquations} (d)) toma valores cercanos a cero, al no producirse ning\'un pico de intensidad. Esto produce que la emisi\'on estimulada no tome valores suficientemente altos como para que la emisi\'on espont\'anea sea despreciable y as\'i, se puede observar en el \chirp\ (Figura \ref{Img:rateEquations} (g)) el ruido debido a la emisión espont\'anea.

		Al aumentar la amplitud de la corriente a $V_{RF} = 1$ V (Figura \ref{Img:rateEquations} (b)) se observa como los aumentos de la corriente durante la oscilaci\'on coinciden con el crecimiento de \n\ (Figura \ref{Img:rateEquations} (k)), haciendo que tome valores muy superiores a $N_{th}$. A su vez, esto produce que, al superar $N(t)$ el valor del umbral $N_{th}$, la \s\ (Figura \ref{Img:rateEquations} (e)) tambi\'en tenga un pico superior al valor del l\'aser en corriente continua. De igual forma que ocurria en el transitorio, al aumentar $S(t)$ y dominar la emisi\'on estimulada, $N(t)$ comienza a disminuir, alcanzando un m\'aximo. Sin embargo, en el momento en el que $N(t)$ alcanza el m\'inimo, la corriente se encuentra por debajo de la corriente umbral $I_{th}$, y $N(t)$ no puede aumentar hasta que $I(t)$ toma nuevamente valores mayores de $I_{th}$. Debido a este tiempo $t$ en el que $N(t)$ no es capaz de volver a aumentar, compensando la disminuci\'on de \s, hay un mayor tiempo $t$ en el que $S(t)$ es cero, y as\'i no hay emisi\'on estimulada. Esta alternancia entre el dominio de la emisi\'on estimulada y la emisi\'on espont\'anea se puede pareciar en el \chirp\ (Figura \ref{Img:rateEquations} (h)), en la que se aprecia el ruido debido a la emisión espont\'anea cuando la densidad de fotones es cero, mienstras que durante los picos de $S(t)$ el ruido es despreciable y no se observa.
			
		Para la amplitud de $V_{RF} = 1.5$ V se observa la misma tendencia que para $V_{RF} = 1$ V, a excepci\'on de que en este caso, al aumentar la amplitud aumenta el tiempo en el que la corriente es menor que $I_{th}$ y as\'i el tiempo en el que $S(t)$ es cero y domina la emisi\'on espontánea.


		En la Figura \ref{Img:PSD} se muestran los espectros de los OFC obtenidos mediante \gs\ para las tres amplitudes de la Figura \ref{Img:rateEquations}.

			% Img:PSD
			\begin{figure}[H]
				\centering
				\includegraphics[width=1.0\linewidth]{PSD.png}
				\caption{\label{Img:PSD}Espectros de los OFC obtenidos mediante \gs\ para $\ibias = 30$ mA, $f_R = 5$ GHz y amplitud de modulaci\'on $V_{RF} = 0.05$ V (verde), $1.00$ V (azul) y $1.50$ V (naranja).}
			\end{figure}
			
		Al igual que se obtuvo en la Figura \ref{Img:rateEquations}, se puede observar como el caso de la amplitud de modulaci\'on $V_{RF} = 0.05$ V se asemeja al del l\'aser en corriente cont\'inua, obteniendo un espectro (Figura \ref{Img:PSD} (verde)) con la frecuencia de emisi\'on dominante de la Figura \ref{Img:spectrosCW}. Como consecuencia del \gs\ realizado se observan excitadas las frecuencias de emisi\'on, apareciendo nuebas l\'ineas de emisi\'on a los lados de la emisi\'on principal.

		Para el caso de $V_{RF} = 1$ V se observa un OFC (Figura \ref{Img:PSD} (azul)) de gran calidad formado por numerosas l\'ineas de emisión equiespaciadas y bien definidas. Se ha obtenido una regi\'on de longitudes de onda con l\'ineas de emisión de la misma densidad espectral de potencia lo cuál resulta enuna gran calidad del OFC.

		Por otro lado, se observa que para el caso de $V_{RF} = 1.5$ V (Figura \ref{Img:PSD} (naranja)) el OFC se detruye debido al ruido de la emisión espont\'anea, obteniendo l\'ineas de emisión poco definidas, con un espaciado variado y mucho ruido.

		De esta forma, se ha podido caracterizar la calidad de los OFC, y del \gs, para altas frecuencias en función de la amplitud de modulaci\'on. Se ha podido observar la creaci\'on del OFC para $V_{RF} = 1$ V, as\'i como la destrucci\'on de este para altas amplitudes, con $V_{RF} = 1.5$ V.

		Tal y como se ha comentado a partir de los resultados de la Figura \ref{Img:rateEquations}, uno de los efectos de aumentar la amplitud de modulaci\'on es la disminuci\'on de la corriente por debajo de $I_{th}$ por un tiempo $t$, que aumenta con la amplitud. Sin embargo, \'esto también se puede controlar para una amplitud fija, variando la corriente de polarizaci\'on $\ibias$.

		En la Figura \ref{Img:current} se muestran la potencia $P(t)$, obtenida a partir de la \s\ \ref{eq:Power}, y los espectros de los OFC con $f_R = 5$ GHz, $V_{RF} = 1$ V e $\ibias = 30$ mA y $50$ mA.

			% Img:current
			\begin{figure}[H]
				\centering
				\includegraphics[width=1.0\linewidth]{current.png}
				\caption{\label{Img:current}Perfil temporal de las potencias $P(t)$ (izquierda) y espectros (derecha) de OFC con $f_R = 5$ GHz, $V_{RF} = 1$ V e $\ibias = 30$ mA (azul) y $50$ mA (naranja).}	
			\end{figure}

		En el perfil temporal de la potencia $P(t)$ de $\ibias = 30$ mA (Figura \ref{Img:current} (izquierda, azul)) se observan las zonas de tiempo con $P(t) \propto S(t) = 0$ vistas en la Figura \ref{Img:rateEquations}, debidas a que la corriente toma valores por debajo de $I_{th}$ y as\'i \n\ no puede aumentar. De igual manera se ha obtenido un OFC de gran calidad (Figura \ref{Img:current} (derecha, azul)) como el obtenido en la Figura \ref{Img:PSD}.

		Sin embargo, en el caso de $\ibias = 50$ mA, al aumentar la corriente de polarizaci\'on, \'esta desplaza la función sinusoidal de la intensidad alejandola de $I_{th}$ y as\'i la amplitud de modulaci\'on no es suficiente para llegar a cruzar $I_{th}$. Esto se puede observar en que el perfil temporal de $P(t)$ (Figura \ref{Img:current} (izquierda, naranja)) no toma nunca el valor cero y as\'i realiza oscilaciones completas. Puesto que la frecuencia $f_R$ y la amplitud $V_{RF}$ de modulación s\'i son suficientes com para que se d\'e \gs, se observa un espectro (Figura \ref{Img:current} (derecha, naranja)) con un OFC formado por l\'ineas bien definidas e igualmente espaciadas. No obstante, el OFC obtenido para $\ibias = 50$ mA es m\'as estrecho que el obtenido para $\ibias = 30$ mA, careciendo de una meseta bien definida con l\'ineas de emisi\'on con densidad espectral de potencia similar. Debido a esto, el OFC obtenido para $\ibias = 30$ mA es de mayor calidad que el obtenido para $\ibias = 50$ mA.

		Otro de los efectos de aumentar la $\ibias$ de tal manera que no cruce la $I_{th}$ es la falta de una regi\'on donde domine la emisi\'on espont\'anea, como ocurria en la Figura \ref{Img:rateEquations}. Esto se puede observar en el menor ruido obtenido en el espectro para $\ibias = 50$ mA (Figura \ref{Img:current} (derecha, naranja)) frente al obtenido en el espectro de $\ibias = 30$ mA (Figura \ref{Img:current} (derecha, azul)).

	\addtocontents{toc}{\vspace{0.1cm}}
	\subsection{Efecto de la amplitud de modulación a bajas frecuencias}
		\label{Sol:OFC:LwFreq}

		Para el estudio del efecto de la amplitud de modulación a bajas frecuencias se ha trabajado con una corriente de polarización $\ibias = 50$ mA y una frecuencia $f_R = 500$ MHz. Se han obtenido la potencia $P(t)$ y los OFC para cuatro amplitudes diferentes, tomando cuatro valores distintos para $V_{RF}$: $0.05$ V, $0.4$ V, $1.0$ V y $1.2$ V.

		En la Figura \ref{Img:500} se muestran los perfiles temporales de la potencia $P(t)$ y los espectros de los OFC para $\ibias = 50$ mA, $f_R = 500$ MHz y $V_{RF} = 0.05$ V, $0.4$ V, $1.0$ V y $1.2$ V.
			% Img:500
			\begin{figure}[H]
				\centering
				\includegraphics[width=1.0\linewidth]{500.png}
				\caption{\label{Img:500}Perfiles temporales de la potencia $P(t)$ (fila superior) y espectros (fila inferior) de los OFC para $\ibias = 50$ mA, $f_R = 500$ MHz y $V_{RF} = 0.05$ V (primera columna), $0.4$ V (segunda columna), $1.0$ V (tercera columna) y $1.2$ V (cuarta columna).}	
			\end{figure}

		Al igual que se obtuvo en el apartado anterior para el caso de altas frecuencias con $\ibias = 50$ mA (Figura \ref{Img:current}), se observan en la Figura \ref{Img:500} perfiles temporales de la potencia oscilantes entorno al valor de la potencia en corriente continua ($P_{CW} \approx 4.7$ mW), variando su amplitud en funci\'on de la amplitud de modulación. Al tener una $\ibias$ muy superior a $I_{th}$ y una frecuencia baja, la amplitud de modulaci\'on no permite mantener la corriente de inyecci\'on por debajo de la corriente umbral un tiempo suficiente como para que $P(t) \propto S(t) = 0$, tal y como se observa en las Figuras \ref{Img:500} (fila superior).

		Para el caso de la amplitud de modulaci\'on pequeña con $V_{RF} = 0.05$ V, se ha obtenido un comportamiento muy similar a la corriente continua, al igual que para altas frecuencias. El espectro obtenido para esta amplitud de modulaci\'on es similar al de la Figura \ref{Img:PSD} (verde) del apartado anterior, obteniendo un pico de emisi\'on dominante correpondiente a la emisi\'on en continua y dos picos a cada lado debidos a la excitaci\'on de la frecuencia de oscilación.

		Se observa como, al igual que ocurria en a altas frecuencias, a medida que aumenta la amplitud de modulación aumenta el n\'umero de l\'ineas de emisión de espectro, llegando a destruirse para altas amplitudes de modulación. Sin embargo, al trabajar a bajar frecuencias se observa una clara irregularidad en el perfil del OFC, tomando los picos valores muy diversos de la densidad espectral de potencia. Esto implica una perdida de la calidad de los OFC a bajas frecuencias con respecto a altas frecuencias. 

		DEBERIA HABLAR SOBRE LOS PICOS QUE SE OBSERVAN EN LA POTENCIA PARA V = 1.2

		En la Figura \ref{Img:500mhz} se muestran los perfiles temporales de la potencia $P(t)$ y los espectros de los OFC para $\ibias = 50$ mA, $f_R = 500$ MHz y $V_{RF} = 0.1$ V, $0.4$ V, $1.2$ V y $1.6$ V obtenidos experimentalmente \cite{Chaves19}.

			% Img:500mhz
			\begin{figure}[H]
				\centering
				\includegraphics[width=1.0\linewidth]{../Chaves/OFC-GS/500mhz.png}
				\caption{\label{Img:500mhz}Perfiles temporales de la potencia $P(t)$ (fila superior) y espectros (fila inferior) de los OFC para $\ibias = 50$ mA, $f_R = 500$ MHz y $V_{RF} = 0.1$ V (primera columna), $0.4$ V (segunda columna), $1.2$ V (tercera columna) y $1.6$ V (cuarta columna) obtenidos experimentalmente \cite{Chaves19}.}	
			\end{figure}

		En la Figura \ref{Img:500mhz} se muestran la potencia $P(t)$ y los espectros experimentales para diferentes amplitudes de modulación, permitiendo comparar los resultados obtenidos mediante simulación con los obtenidos en el laboratorio.

		En la cuarta columna de la Figura \ref{Img:500mhz} se muestran los resultados obtenidos para una amplitud de $V_{RF} = 1.6$ V, observando como el espectro de frecuencia se encuentra completamente destruido. Dichos resultados no pudieron ser obtenidos mediante la simulación debido al problema en el transitorio con $\sqrt{S(t)}$. En la primera columna de la Figura \ref{Img:500mhz} se muestran los resultados para una amplitud de modulación pequeña de $V_{RF} = 0.1$ V. Al igual que en los resultados de la simulación, se obtiene un comportamiento similar al de la corriente continua con un pico de emisi\'on dominante. No obstante, al tratarse de el doble de la amplitud utilizada en la simulaci\'on de la Figura \ref{Img:500} se obtienen un mayor n\'umero de picos de emisión estimulados. Los resultados que se muestran en la segunda columna de la Figura \ref{Img:500mhz}  para $V_{RF} = 0.4$ V equivalen a los resultados de la simulación de la segunda columna de la Figura \ref{Img:500}. En ambas figuras se pueden observar un OFC con un perfil aproximandamente sim\'etico con un pico de menor intensidad en el centro. Por \'ultimo, la tercera columna de la Figura \ref{Img:500mhz} equivale a la cuarta columna de la Figura \ref{Img:500}. Ambos espectros presentan un perfil similar con un aumento brusco de la densidad espectral de potencia de los picos para bajas longitudes de onda, seguido de una disminuci\'on m\'as tenue para lontitudes de onda mayores. Ambos perfiles temporales de potencia presentan LOS PICOS DE LOS QUE HE DE HABLAR.

	El excelente acuerdo entre los resultados de la simulaci\'on de la Figura \ref{Img:500} y los resultados experimentales de la Figura \ref{Img:500mhz} indican la capacidad de la simulaci\'on de explicar los procesos que involucra el \gs\ en la generaci\'on de OFC, pudiendo servir para la caracterizaci\'on de la calidad de los OFC.
	

			\addtocontents{toc}{\vspace{0.1cm}}
			\chapter{Inyeccion de Luz}

				\graphicspath{{../Graphics/Cpt2-InjectCW/}}

En este c\'apitulo se ha estudiado la creci\'on de OFC mediante inyecci\'on de luz en funci\'on de las condiciones de la inyecci\'on, dadas por la potencia inyectada $P_{Iny}$ y la diferencia de frecuencias $\delta \nu$ entre la frecuencia del l\'aser maestro y el l\'aser esclavo (ML y SL respectivamente por sus siglas en ingl\'es). Se ha trabajado con el l\'aser ML en corriente continua con $\ibias = 35$ mA, $V_{RF} = 0$ V y $f_R = 5.0$ GHz.

Se han obtenido los diferentes r\'egimenes din\'amicos de los OFC en funci\'on de $P_{Iny}$ para dos valores de $\delta\nu$ distintos, uno positivo que equivale a una frecuencia de SL menor que la de ML, y otro negativo con el caso contrario.

En la Figura \ref{Img:zonasIO} se muestran los espectros \'opticos de las diferentes regiones din\'amicas obtenidas en funci\'on de $P_{Iny}$ para $\delta\nu = -2$ GHz. Se indica la frecuencia de inyecci\'on $\nu_{SL}$ con una flecha. 

			\begin{figure}[H]
				\centering
				\includegraphics[width=1.0\linewidth]{zoneMap.png}
				\caption{\label{Img:zonasIO}Espectros \'opticos de las diferentes regiones din\'amicas obtenidas en funci\'on de $P_{Iny}$ para $\delta\nu = -2$ GHz. Se indica la frecuencia de inyecci\'on $\nu_{SL}$ con una flecha.}
			\end{figure}

		Para una baja potencia de inyecci\'on $P_{Iny} = 1 \; \mu$W (Figura \ref{Img:zonasIO} (a)) se obtiene un espectro \'optico con el pico de emisi\'on de ML y dos picos estimulados para la frecuencia de inyecci\'on $\nu_{SL}$. En esta regi\'on denominada de periodo 1, P1, las variables internas del \'aser logran estabilizarse \cite{vainio2006diode} debido a un fen\'omeno de \textit{Four-wave mixing} (FWM, mezcla de cuatro ondas). Al aumentar $P_{Iny}$ se llega a una regi\'on de caos, CH-IR ($P_{Iny} = 20\; \mu$W, Figura \ref{Img:zonasIO} (b)), con un OFC formado por muchas l\'ineas y con un perfil irregular. Esta regi\'on de caos se destruye para $P_{Iny} = 100\;\mu$W, en la que se obtiene un espectro \'optico con una \'unica l\'inea de emisi\'on para la frecuencia de inyecci\'on $\nu_{SL}$. Al aumentar la potencia de SL el l\'aser bloque la emisi\'on en $\nu_{ML}$ pasando a emitir solo en $\nu_{SL}$. A este fen\'omeno se le conoce como Bloqueo de Inyecci\'on (IL por sus siglas en ingl\'es). En la Figura \ref{Img:zonasIO} (d) se muestra el espectro para $P_{Iny} = 200\;\mu$W en IL, observando com al aumentar la potencia de inyecci\'on se empiezan a estimular las frecuencias de las oscilaciones de relajaci\'on. Esto ind\'ica que nos encontramos en el l\'imite de la regi\'on IL, encontrando una bifurcaci\'on de Van't Horf. Si se contin\'ua aumentando la potencia de inyecci\'on aumentar\'an los picos de la frecuencia de oscilaciones de relajaci\'on, retornando a la regi\'on P1 ($V_{RF} = 300\;\mu$W, Figura \ref{Img:zonasIO} (e)). Dentro de la refi\'on P1, el aumento de la potencia de inyecci\'on produce la aparici\'on de nuevas l\'ineas de emisi\'on, creciendo el OFC. Para altas potencias de inyecci\'on, $P_{Iny} = 8000 \;\mu$W y $20000 \;\mu$W, se regresa a la regi\'on IL.

		A partir de los datos experimentales \cite{Chaves19} se ha obtenido un mapa con las diferentes regiones din\'amicas en funci\'on de $P_{Iny}$ y $\delta\nu$. En la Figura \ref{fig:map} se muestra el mapa de las regiones din\'amicas obtenido a partir de \cite{Chaves19}, marcando los putos correspondientes a los espectros \'opticos de la Figura \ref{Img:zonasIO}.

			\begin{figure}[H]
				\centering
				\includegraphics[width=0.7\linewidth]{maps.png}
				\caption{\label{fig:map}Mapa con las diferentes regiones din\'amicas en funci\'on de $P_{Iny}$ y $\delta\nu$ obtenido a partir de \cite{Chaves19}. Se han marcando los putos correspondientes a los espectros \'opticos de la Figura \ref{Img:zonasIO}.}
			\end{figure}

		Las regiones din\'amicas del mapa de la Figura \ref{fig:map} obtenidas experimentalmente, para las condiciones de inyecci\'on de los espectros \'opticos de la Figura \ref{Img:zonasIO} corresponden a las regiones obtenidas del an\'alisis de los resultados de la simulaci\'on.

		En la Figura \ref{fig:zoneRtEq} se muestran la potencia $P(t)$, la fase \'optica $\Phi (t)$ y el espectro \'optico de los tres casos m\'as representativos de la Figura \ref{Img:zonasIO} para cada regi\'on din\'amica obtenida: CH-IR, IL y P1. Esto permite estudiar los procesos que tienen lugar en las tres regiones encontradas para $\delta\nu = -2$ GHz. 

			\begin{figure}[H]
				\centering
				\includegraphics[width=1.0\linewidth]{zoneRtEq.png}
				\caption{\label{fig:zoneRtEq}Potencia $P(t)$, fase \'optica $\Phi (t)$ y espectro \'optico de los tres casos m\'as representativos de la Figura \ref{Img:zonasIO} para cada regi\'on din\'amica obtenida: CH-IR con $P_{Iny} = 20\;\mu$W (verde), IL con $P_{Iny} = 100\;\mu$W (azul) y P1 con $P_{Iny} = 1000\;\mu$W (naranja). Se indica en los espectros la frecuencia de inyecci\'on $\nu_{SL}$ con una flecha.}	
			\end{figure}

		Para el caso con $P_{Iny} = 1000\;\mu$W de la rigi\'on P1 se obtiene un espectro \'optico (Figura \ref{fig:zoneRtEq} (c)) se obtiene un OFC de buena calidad formado por varias l\'ineas bien resueltas y con las misma serparaci\'on entre ellas. Los perfiles temporales que se obtienen para $P(t)$ y $\Phi(t)$ son oscilaciones con una amplitud y periodo bien definidos, mostrando la estabilizaci\'on debida al FWM. En la Figura \ref{fig:zoneRtEq} (e)  se muestra el perfil temporal de la potencia para $P_{Iny} = 100\;\mu$W IL, que toma valores cercanos a un valor constante, realizando variaciones aleatroria y pequeñas entorno a dicho valor. Esto mismo se observa para su fase \'optica (Figura \ref{fig:zoneRtEq} (h)) en la que se obtienen variaciones de $\Phi(t)$ tres ordenes de magnitud menores que para P1. Con $P_{Iny} = 20\;\mu$W se encuentra la regi\'on CH-IR, con un perfil temporal de $P(t)$ similar al de IL (Figura \ref{fig:zoneRtEq} (d)) pero con unas pequeñas oscilaciones anarm\'onicas en un periodo $T \approx 1.25$ ns. La fase \'otica en CH-IR (Figura \ref{fig:zoneRtEq} (g)) aumenta a medida que el tiempo avanza.

		Del mapa de regiones din\'amicas de la Figura \ref{fig:map} se deduce que para $\delta\nu$ positivo se ha de poder alcanzar regiones con doblamiento de periodo, P2. En la Figura \ref{fig:P2zone} se muestran los espectros \'opticos, $P(t)$ y el atractor en el espacio de estados de las ecuaciones de balance, despreciando los efectos de la fase \'optica; para $\delta\nu = 5$ GHz y $P_{Iny} = 50\;\mu \textrm{W, } 1000\;\mu\textrm{W y } 200\;\mu$W.

			\begin{figure}[H]
				\centering
				\includegraphics[width=1.0\linewidth]{P2zone.png}
				\caption{\label{fig:P2zone}Espectros \'opticos, $P(t)$ y atractor en el espacio de estados de las ecuaciones de balance, despreciando los efectos de la fase \'optica; para $\delta\nu = 5$ GHz y $P_{Iny} = 50\;\mu \textrm{W (verde), } 1000\;\mu\textrm{W (azul) y } 200\;\mu$W (naranja). Se indica en los espectros la frecuencia de inyecci\'on $\nu_{SL}$ con una flecha.}	
			\end{figure}

			Se obtiene de nuevo un OFC de buena calidad para la regi\'on P1 ($P_{Iny} = 50\;\mu$W, Figura \ref{fig:P2zone} (a)) y una $P(t)$ oscilante con una amplitud y frecuencia determinada (Figura \ref{fig:P2zone} (b)). Sin embargo, se observa como los m\'aximos de $P(t)$ comienzan a desdoblarse formando una segunda oscilaci\'on, indicando que se encuentra cerca de una regi\'on con doblamiento de periodo P2. \'Esto no se observa en el espectro \'optico debido a la poca intensidad de estos picos y al ruido debido a la emisi\'on espont\'anea, pero s\'i se puede ver en la Figura \ref{fig:P2zone} (c). El diagrama no llega a realizar una revoluci\'on completa sino que se dobla hacia el interior en un determinado punto. Al llegar la regi\'on P2 ($P_{Iny} = 1000\;\mu$W) la potencia se ha desdoblado completamente (Figura \ref{fig:P2zone} (e)), alternando picos de mayor y menor potencia. En el espectro \'optico (Figura \ref{fig:P2zone} (d)) se obtienen l\'ineas de emisi\'on escitadas entre las que se te\'ian en P1. La frecuencia de separaci\'on entre las l\'ineas cae a la mitad $\Delta \nu' = \frac{\Delat\nu}{2}$ y as\'i el periodo es el doble. En la Figura \ref{fig:P2zone} (f) aparece una nueva oscilaci\'on de menor amplitud debido al desdoblamiento de $P(t)$ y $N(t)$.

		Se alcanza la regi\'on CH-IR para $P_{Iny} = 200\;\mu$W, obteniendo un perfil de $P(t)$ con oscilaciones aleatrorias y sin una amplitud o frecuencia determinada (Figura \ref{fig:P2zone} (h)).El OFC del espectro \'optico (Figura \ref{fig:P2zone} (g)) se destruye completamente y el diagrama de estados de la Figura \ref{fig:P2zone} (i) describe una trayectoria irregular que para rangos de tiempo suficientemente grandes cubrir\'ia todo el espacio. 

		En la Figura \ref{fig:maps2} se muestra el mapa de las regiones din\'amicas obtenido a partir de \cite{Chaves19}, marcando los puntos correspondientes a las condiciones de inyecci\'on de los resultados de la Figura \ref{fig:P2zone}, obteniendo las mismas regiones que mediante el an\'alisis de los resultados de la simulaci\'on.

			\begin{figure}[H]
				\centering
				\includegraphics[width=0.7\linewidth]{maps2.png}
				\caption{\label{fig:maps2}Mapa con las diferentes regiones din\'amicas en funci\'on de $P_{Iny}$ y $\delta\nu$ obtenido a partir de \cite{Chaves19}. Se han marcando los putos correspondientes a las condiciones de inyecci\'on de la Figura \ref{fig:P2zone}.}	
			\end{figure}


				
			\addtocontents{toc}{\vspace{0.1cm}}
			\chapter{Inyeccion de luz en OFC}

				\graphicspath{{../Graphics/Cpt3-CombInject/}}


			\begin{figure}[H]
				\centering
				\includegraphics[width=1.0\linewidth]{psdMap.png}
				\caption{\label{Img:widgets}el pie de pagina que le quieras 	poner a la imagen}
			\end{figure}

			\begin{figure}[H]
				\centering
				\includegraphics[width=1.0\linewidth]{p1-p2.png}
				\caption{\label{fig:p1-p2}p1-P2}	
			\end{figure}

			\begin{figure}[H]
				\centering
				\includegraphics[width=1.0\linewidth]{chaos.png}
				\caption{\label{fig:chaos}Chaos}	
			\end{figure}

%
				%\cite{rosado2018experimental}
%
			\addtocontents{toc}{\vspace{0.1cm}}
			\chapter{Conclusiones}

				hola a todos

		%----------------------------------------------------------------------------------------
		%     BIBLIOGRAPHY
		%----------------------------------------------------------------------------------------

			\bibliographystyle{unsrt}
			\bibliography{biblio}

		%----------------------------------------------------------------------------------------
		%     APPENDIX
		%----------------------------------------------------------------------------------------
				
			\newpage

				\appendix

					\chapter{Código de la simulación}

						El c\'odigo utilizado para la simulaci\'on de la din\'amica del l\'aser de semiconductor \gs\ se ha desarrollado utilizando el lenguaje de programaci\'on Python, versi\'on python 2.7. Los diferentes scripts utilizados para la simulaci\'on \cite{github} permiten resolver las ecuaciones de balance \ref{eq:RtEq-N}-\ref{eq:RtEq-Ph}.

Para el c\'alculo de las ecuaciones de balance se ha utilizado el m\'etodo de resoluci\'on de ecuaciones estoc\'asticas descrito en la secci\'on \ref{Intr:PrcsEstcs}, expandiento las ecuaciones \ref{eq:RtEq-N}-\ref{eq:RtEq-Ph} con las expresiones  \ref{eq:MatGain} \ref{eq:CarriRcom} y \ref{eq:gainSwtching}. Se han incluido tambi\'en los t\'erminos de la inyecci\'on \'optica \ref{eq:Iny-S} y \ref{eq:Iny-Phi}.

	\begin{equation}
		\begin{matrix}
			N(t + \mathrm{d}t) =  & N(t) + \frac{\Delta t \ibias }{e V_{act}} + \frac{\Delta t C_{loss} 2 \sqrt{2} V_{RF}}{eV_{act} (Z_0+Z_l)} \sin(2\pi f_R t) \\ \\
			 & - A\Delta t N(t) - B\Delta t N(t)^2 - C\Delta t N(t)^3 \\ \\
			 & - v_g \frac{\mathrm{d}g}{\mathrm{d}N} \Delta t N(t) \frac{1}{1/S(t) + \epsilon}  + v_g \frac{\mathrm{d}g}{\mathrm{d}N} \Delta t N_{tr} \frac{1}{1/S(t) + \epsilon}
		\end{matrix}
		\label{eq:Code-N}
	\end{equation}

	\begin{equation}
		\begin{matrix}
			S(t + \mathrm{d}t) =  & S(t) + \Gamma v_g \frac{\mathrm{d}g}{\mathrm{d}N} \Delta t N(t) \frac{1}{1/S(t) + \epsilon} - \Gamma v_g \frac{\mathrm{d}g}{\mathrm{d}N} \Delta t N_{tr} \frac{1}{1/S(t) + \epsilon} \\ \\
			 & - \frac{\Delta t}{\tau_p}S(t) + \beta\Gamma B\Delta t N(t)^2 + \sqrt{2 \beta \Gamma B \Delta tN^2(t)S(t)} X_i \\ \\
			 & + 2k_c\sqrt{S(t)S_{Iny}} \cos(\Phi(t) - 2\pi \delta\nu't)
		\end{matrix}
		\label{eq:Code-S}
	\end{equation}

	\begin{equation}
		\begin{matrix}
			\Phi(t + \mathrm{d}t) =  & \Phi(t) + \frac{\alpha}{2}\Gamma v_g \frac{\mathrm{d}g}{\mathrm{d}N} \Delta t N(t) - \frac{\alpha}{2}\Gamma v_g \frac{\mathrm{d} g}{\mathrm{d}N} N_{tr} - \frac{\alpha\Delta t}{2\tau_p} + 2\pi\Delta\nu(I)\Delta t \\ \\
			 & + \sqrt{\frac{\beta \Gamma B \Delta t N^2(t)}{2 S(t))}} Y_i - k_c\sqrt{\frac{S_{Iny}}{S(t)}} \sin(\Phi(t) - 2\pi \delta\nu't)
		\end{matrix}
		\label{eq:Code-Ph}
	\end{equation}

En las ecuaciones \ref{eq:Code-N}-\ref{eq:Code-Ph} aparecen: $\Delta t$ el tiempo de integraci\'on, $\Delta\nu(I)$ la diferencia de frecuencias entre la frecuencia de emisi\'on a $\ibias$ y la frecuencia de emisi\'on en la corriente umbral \cite{Chaves19} y los t\'erminos de ruido gaussiano $X_i$ e $Y_i$ con $N(0, 1)$ e independientes entre s\'i. Para estos t\'erminos de ruido gaussiano $X_i$ e $Y_i$ se ha utlizado la funci\'on \texttt{numpy.random.normal(loc=0, scale=1, size=nTotal)} de la libreria \texttt{NumPy} para Python \cite{numpy}.

Apartir de estas ecuaciones se han obtenido diferentes t\'erminos que no dependen del tiempo (independientes de $N(t)$ y $S(t)$), permitiendo ser calculados antes de la ejecuci\'on de la simulaci\'on. \'Esto se realiza en el script \texttt{Constants.py}, que es importado al realizar la simulaci\'on, ahorrando tiempo de computaci\'on. Con \'este objetivo tambi\'en se han desarrollado las funciones seno y coseno de los t\'erminos de la inyecci\'on teniendo en cuenta las propiedades de esta\'as funciones para resta de \'angulos.

	\begin{equation}
		\begin{matrix}
			\sin(u - v) = \sin(u)\cos(v) - \cus(u)\sin(v) \\ \\

			\cos(u - v) = \cos(u)\cos(v) + \sin(u)\sin(v) 
		\end{matrix}
	\end{equation}

	Los t\'erminos de la inyecci\'on \'optica $Y_S$ e $Y_{\Phi}$ de las ecuaciones \ref{eq:Iny-S} y \ref{eq:Iny-Phi} vienen ccaracterizados por $S_{Iny}$ y $\delta\nu'$. Sin embargo, para un mayor entendimiento en la comparaci\'on con los resultados experimentales, la inyecci\'on \'optica en la simulaci\'on ha sido caracterizada por su potencia inyectada $P_{Iny}$, pudiendo obtener $S_{Iny}$ con la ecuaci\'on \ref{eq:Power}, y por la diferencia de frecuencias $\delta\nu$ entre la frecuencia de inyecci\'on del l\'aser maestro $\nu_{ML}$ y la frecuencia de emisi\'on del l\'aser esclavo sin \gs\ $\nu_{SL}$, que depende de la corriente $\ibias$. Puesto que $\delta\nu'$ viene definido por la frecuencia de emisi\'on del l\'aser esclavo en el umbral $\nu_{th}$, es necesario realizar un cambio de variable.

	\begin{equation}
		\delta\nu' = \delta\nu - \nu_{th} + \nu
	\end{equation}

	Tambi\'en se observa en la ecuaci\'on \ref{eq:Code-N} como la modulaci\'on de la corriente $I(t)$ solo depende del tiempo, por lo que $\sin(2\pi f_R t)$ puede ser calculado y almacenado en un vector al comienzo de la simulaci\'on, ahorrando tiempo de c\'alculo. Se ha considerado $C_{loss} = 1$ para simplificar los c\'alculos \cite{Chaves19}.

	En \texttt{Constants.py} se computan, junto con los t\'erminos nindependientes de $N(t)$ y $S(t)$ de las ecuaciones \ref{eq:Code-N}-\ref{eq:Code-Ph}, el resto de par\'ametros de dichas ecuaciones necesarios para la realizaci\'on de la simulaci\'on, obtenidos de \cite{artSim} y \cite{Chaves19}.

	Puesto que en las ecuaciones \ref{eq:Code-N}-\ref{eq:Code-Ph} se trabaja en referencia a una corriente umbral y una frecuencia umbral, es necesario tener en cuenta los cambios de determinados t\'erminos con la corriente de polarización $\ibias$. En el script \texttt{getDictValues.py} se inicializan los diferentes valores de la frecuencia de emisión $\nu$, la diferencia de frecuencias respecto a la frecuencia de emisi\'on en la corriente umbral \cite{Chaves19} y el corrimiento de frecuencias de la transformada r\'apida de Fourier; en diccionarios de python.

	\addtocontents{toc}{\vspace{0.1cm}}
	\subsection{Transformada R\'apida de Fourier}
		\label{Mdl:Code:Temp}

	\addtocontents{toc}{\vspace{0.1cm}}
	\subsection{T\'ermino de la temperatura}
		\label{Mdl:Code:Temp}

		Explicar el termino de la temperatura

	\addtocontents{toc}{\vspace{0.1cm}}
	\subsection{Transitorio}
		\label{Mdl:Code:Trans}

		Explicar el transitorio


	\end{document}

%----------------------------------------------------------------------------------------
%              END
%----------------------------------------------------------------------------------------
