\graphicspath{{../Graphics/Cpt2-InjectCW/}}

Una forma alternativa de obtención de OFC es mediante la inyección de luz láser en otro láser de semiconductor funcionando en corriente constante. Sin embargo este sistema presenta una dinámica no lineal muy variada aparte de la observación de los OFC. En este cápitulo se han estudiado los distintos comportamientos no lineales observados en este sistema en función de las condiciones de la inyección óptica, dadas por la potencia inyectada $P_{Iny}$ y la diferencia de frecuencias $\delta \nu$ entre el láser maestro y el láser esclavo (ML y SL respectivamente por sus siglas en inglés). Se ha trabajado con el láser SL en corriente continua con $\ibias = 35$ mA, $V_{RF} = 0$ V y $f_R = 5.0$ GHz.

Se han obtenido los diferentes régimenes dinámicos de SL en función de $P_{Iny}$ para dos valores de $\delta\nu$ distintos, uno positivo que equivale a una frecuencia de SL menor que la de ML, y otro negativo con el caso contrario.

En la Figura \ref{Img:zonasIO} se muestran los espectros ópticos del láser esclavo con inyecci\'on \'optica de las diferentes regiones dinámicas obtenidas para diferentes valores de $P_{Iny}$ a $\delta\nu = -2$ GHz. 

			\begin{figure}[H]
				\centering
				\includegraphics[width=1.0\linewidth]{zoneMap.png}
				\caption{\label{Img:zonasIO}Espectros ópticos con inyección \'optica de las diferentes regiones dinámicas obtenidas para diferentes valores de $P_{Iny}$ para $\delta\nu = -2$ GHz. Se indica la frecuencia de inyección $\nu_{ML}$ con una flecha y $P_{Iny}$ para cada espectro óptico.}
			\end{figure}

		Para una baja potencia de inyección $P_{Iny} = 1 \; \mu$W (Figura \ref{Img:zonasIO} (a)) se obtiene un espectro óptico con el pico de emisión de SL y tres picos satélites, uno de ellos apareciendo a $\nu_{ML}$ y los otros simétricos respecto al pico principal. En esta región denominada de periodo 1, P1, las variables internas del láser tienen un comportamiento periodico. Cuando las potencias de inyección son bajas, la frecuencia de este comportamiento es $\delta\nu$ debido a un fenómeno de \textit{Four-wave mixing} (FWM, mezcla de cuatro ondas) \cite{van1995semiconductor}. Al aumentar $P_{Iny}$ se llega a una región de caos, CH-IR ($P_{Iny} = 20\; \mu$W, Figura \ref{Img:zonasIO} (b)), con un OFC formado por muchas líneas y con un perfil irregular. Esta región de caos se destruye para $P_{Iny} = 100\;\mu$W, en la que se obtiene un espectro óptico con una única línea de emisión para la frecuencia de inyección $\nu_{ML}$. Al aumentar la potencia de ML el láser deja de emitir en $\nu_{SL}$ pasando a emitir solo en $\nu_{ML}$. A este fenómeno se le conoce como Bloqueo de Inyección (IL por sus siglas en inglés). El régimen IL se caracteriza además porque el láser esclavo pasa a emitir con una fase óptica relativa a la del láser maertro con un valor constante. Addemás la potencia del láser esclavo es constante. En la Figura \ref{Img:zonasIO} (d) se muestra el espectro para $P_{Iny} = 200\;\mu$W en IL, observando como al aumentar la potencia de inyección se empiezan a excitarlos picos satélites correspondicentes a las oscilaciones de relajación. Esto indíca que nos encontramos en el límite de dos comportamientos, el descrito para la región IL y un comportamiento periódico, con periodo el de las oscilaciones de relajación (este cambio de comportamientos corresponde a una bifurcación de Hopf. Si se continúa aumentando la potencia de inyección aumentarán los picos de la frecuencia de oscilaciones de relajación, retornando a la región P1 ($V_{RF} = 300\;\mu$W, Figura \ref{Img:zonasIO} (e)). Dentro de la región P1, el aumento de la potencia de inyección produce la aparición de nuevas líneas de emisión, comenzando a aparecer el OFC. A medida que aumenta la potencia de inyección la separación entre líneas consecutivas del espectro óptico va creciendo  PONER LA SEPARACION ENTRE LINEAS PARA 300 Y 1000. Para altas potencias de inyección, $P_{Iny} = 8000 \;\mu$W y $20000 \;\mu$W, se regresa a la región IL.

		A partir de los datos experimentales para un láser de modo discreto en corriente continua $\ibias = 30$ mA \cite{Chaves19}, se ha obtenido un mapa con las diferentes regiones dinámicas en función de la potencia inyectada experimental $P_{Inj}$ y $\delta\nu$. Hacemos notar que la potencia inyectada teórica $P_{Iny}$ es siempre mayor que la potencia inyectada experimental $P_{Inj}$ debido a las pérdidas que sufre la luz del láser maestro en el experimento antes de ser inyectado en el láser esclavo. Se ha estimado el término de proporcionalidad entre ambas potencias $P_{Inj} = 0.077 \cdot P_{Iny}$ \cite{Chaves19}. Aunque la corrietnte teórica es diferente de la experimental la comparación es adecuada porque en ambos casos la relación entre la corriente y la corriente umbral es similar $\frac{\ibias}{I_{th}} \approx 2.3$. En la Figura \ref{fig:map} se muestra el mapa de las regiones dinámicas obtenido a partir de \cite{Chaves19}, marcando los puntos correspondientes a los espectros ópticos de la Figura \ref{Img:zonasIO}.

			\begin{figure}[H]
				\centering
				\includegraphics[width=0.7\linewidth]{maps.png}
				\caption{\label{fig:map}Mapa con las diferentes regiones dinámicas en función de $P_{Iny}$ y $\delta\nu$ obtenido a partir de \cite{Chaves19}. Se han marcando los puntos correspondientes a los espectros ópticos de la Figura \ref{Img:zonasIO}.}
			\end{figure}

		Las regiones dinámicas del mapa de la Figura \ref{fig:map} obtenidas experimentalmente, para las condiciones de inyección de los espectros ópticos de la Figura \ref{Img:zonasIO} corresponden a las regiones obtenidas del análisis de los resultados de la simulación. Lo cuál indica un buen acuerdo entre teoría y experimento.

		En la Figura \ref{fig:zoneRtEq} se muestran la potencia $P(t)$, la fase óptica $\Phi (t)$ y el espectro óptico de los tres casos más representativos de la Figura \ref{Img:zonasIO} para cada región dinámica obtenida: CH-IR, IL y P1. Esto permite estudiar los procesos que tienen lugar en las tres regiones encontradas para $\delta\nu = -2$ GHz. 

			\begin{figure}[H]
				\centering
				\includegraphics[width=1.0\linewidth]{zoneRtEq.png}
				\caption{\label{fig:zoneRtEq}Potencia $P(t)$, fase óptica $\Phi (t)$ y espectro óptico de los tres casos más representativos de la Figura \ref{Img:zonasIO} para cada región dinámica obtenida: CH-IR con $P_{Iny} = 20\;\mu$W (verde), IL con $P_{Iny} = 100\;\mu$W (azul) y P1 con $P_{Iny} = 1000\;\mu$W (naranja). Se indica en los espectros la frecuencia de inyección $\nu_{ML}$ con una flecha.}	
			\end{figure}

		Para el caso con $P_{Iny} = 1000\;\mu$W de la región P1 se obtiene un espectro óptico (Figura \ref{fig:zoneRtEq} (c)) se obtiene un OFC de buena calidad formado por varias líneas bien resueltas y con las misma separación entre ellas (PONER SEPARACION GHz). Los perfiles temporales que se obtienen para $P(t)$ y $\Phi(t)$ son oscilaciones con una amplitud y una frecuecia bien definidos con $VALOR DE LA FRECUENCIA ANTERIOR$. En la Figura \ref{fig:zoneRtEq} (e)  se muestra el perfil temporal de la potencia para $P_{Iny} = 100\;\mu$W IL, que toma Uun valor aproximadamente constante. Esto mismo se observa para su fase óptica (Figura \ref{fig:zoneRtEq} (h)) en la que se obtienen variaciones de $\Phi(t)$ tres ordenes de magnitud menores que para P1. Con $P_{Iny} = 20\;\mu$W se encuentra la región CH-IR, VER COMO EXPLICAR QUE LAS TRAZAS SON IRREGULARES PARA TIEMPOS MAS GRANDES.

		Del mapa de regiones dinámicas de la Figura \ref{fig:map} se deduce que para $\delta\nu$ positivo se ha de poder alcanzar regiones con doblamiento de periodo, P2. En la Figura \ref{fig:P2zone} se muestran los espectros ópticos, $P(t)$ y el atractor en el espacio de estados de las ecuaciones de balance, despreciando los efectos de la fase óptica; para $\delta\nu = 5$ GHz y $P_{Iny} = 50\;\mu \textrm{W, } 1000\;\mu\textrm{W y } 200\;\mu$W.

			\begin{figure}[H]
				\centering
				\includegraphics[width=1.0\linewidth]{P2zone.png}
				\caption{\label{fig:P2zone}Espectros ópticos, $P(t)$ y atractor en el espacio de estados de las ecuaciones de balance, despreciando los efectos de la fase óptica; para $\delta\nu = 5$ GHz y $P_{Iny} = 50\;\mu \textrm{W (verde), } 1000\;\mu\textrm{W (azul) y } 200\;\mu$W (naranja). Se indica en los espectros la frecuencia de inyección $\nu_{ML}$ con una flecha.}	
			\end{figure}

		REESCRIBIR EL PARAFO ANALIZANDO LOS CASOS P1 Y P2 POR SEPARADO, REALIZANDO LA COMPARACION SOLO PARA EL ATRACTOR.

		Se obtiene de nuevo un OFC de buena calidad para la región P1 ($P_{Iny} = 50\;\mu$W, Figura \ref{fig:P2zone} (a)) y una $P(t)$ oscilante con una frecuencia de 4 GHz (Figura \ref{fig:P2zone} (b)). Sin embargo, se observa como los máximos de $P(t)$ comienzan a desdoblarse formando una segunda oscilación, indicando que se encuentra cerca de una región con doblamiento de periodo P2. Ésto no se observa en el espectro óptico debido a la poca intensidad de estos picos y al ruido debido a la emisión espontánea, pero sí se puede ver en la Figura \ref{fig:P2zone} (c). El diagrama no llega a realizar una revolución completa sino que se dobla hacia el interior en un determinado punto. Al llegar la región P2 ($P_{Iny} = 1000\;\mu$W) la potencia se ha desdoblado completamente (Figura \ref{fig:P2zone} (e)), alternando picos de mayor y menor potencia. En el espectro óptico (Figura \ref{fig:P2zone} (d)) se obtienen líneas de emisión escitadas entre las que se teían en P1. La frecuencia de separación entre las líneas cae a la mitad $\Delta \nu' = \frac{\Delta\nu}{2}$ y así el periodo es el doble. En la Figura \ref{fig:P2zone} (f) aparece una nueva oscilación de menor amplitud debido al desdoblamiento de $P(t)$ y $N(t)$.

		Se alcanza la región CH-IR para $P_{Iny} = 200\;\mu$W, obteniendo un perfil de $P(t)$ con oscilaciones aleatrorias y sin una amplitud o frecuencia determinada (Figura \ref{fig:P2zone} (h)). El OFC del espectro óptico (Figura \ref{fig:P2zone} (g)) se destruye completamente y el diagrama de estados de la Figura \ref{fig:P2zone} (i) describe una trayectoria irregular que para rangos de tiempo suficientemente grandes cubriría todo el espacio. 

		En la Figura \ref{fig:maps2} se muestra el mapa de las regiones dinámicas obtenido en \cite{Chaves19}, marcando los puntos correspondientes a las condiciones de inyección de los resultados de la Figura \ref{fig:P2zone}, obteniendo las mismas regiones que mediante el análisis de los resultados de la simulación.

			\begin{figure}[H]
				\centering
				\includegraphics[width=0.7\linewidth]{maps2.png}
				\caption{\label{fig:maps2}Mapa con las diferentes regiones dinámicas en función de $P_{Iny}$ y $\delta\nu$ obtenido a partir de \cite{Chaves19}. Se han marcando los puntos correspondientes a las condiciones de inyección de la Figura \ref{fig:P2zone}.}	
			\end{figure}


ATRACTOR ---> proyección  del atractor en el plano (P, N) del espacio de estados