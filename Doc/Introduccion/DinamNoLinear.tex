
	%En esta seccion se hablara de la dinamica no lineal, bifurcaciones, M.T., y de las dos formas de las que se puede obtener dinamica no lineal en nuestro laser de semiconductor

Los sistemas din\'amicos son aquellos que pueden ser descritos mediante ecuaciones diferenciales ordinarias $\frac{\mathrm{d}\vec{x}}{\mathrm{d}t} = \vec{f}(\vec{x})$ con $\vec{x}$ continua en el espacio. Estos sistemas son a su vez sistemas deterministas, pues es posible, en principio, predecir su comportamiento futuro a partir del conocimiento preciso de su estado en un tiempo $t$ anterior. Las variables $\vec{x} = (x_1,..., x_p)$ forman el espacio de estados, que para un sistema mec\'anico es el espacio de fases con dimensi\'on $p = 2n$, siendo $n$ el n\'umero de grados de libertad.

El estudio de los sistemas din\'amicos cuyas ecuaciones no evolucionan con un comportamiento lineal se denomina din\'amica no lineal. Estos sistemas pueden realizar un comportamiento casi aleatorio y err\'atico, obteniendo caos determinista. Los sistemas ca\'oticos tambi\'en se caracterizan por su gran sensibilidad a las condiciones iniciales, o \textit{divergencia de trayectorias cercanas}. Esta divergencia puede producir diferencias muy notables al cabo de un cierto tiempo en las trayectorias de dos sistemas con condiciones iniciales muy similares, obteniendo un comportamiento impredecible del sistema. Sin embargo, los sistemas ca\'oticos se consideran deterministas en el sentido de que es posible predecir su comportamiento futuro conociendo las condiciones iniciales exactas, a partir de la integraci\'on de las ecuaciones diferenciales del sistema.

Sin embargo, en los experimentos f\'isicos reales es casi imposible recrear las condiciones iniciales con suficiente exactitud. Es por ello que cobra mayor importancia el concepto de estabilidad din\'amica, definiendo \'esta como la sensibilidad de una soluci\'on, a pequeñas variaciones de las condiciones iniciales. Si se tienen dos trayectorias $x(t)$ e $y(t)$ para las cuales en un instante dado $t_0$, $x(t_0)$ e $y(t_0)$ est\'an suficientemente cerca, $x(t)$ ser\'a uniformemente estable en el sentido de \textit{Lyapunov} si $x(t)$ e $y(t)$ se mantienen pr\'oximos en todo el tiempo.

Los sistemas en los cuales el volumen f\'asico, o volumen en el espacio f\'asico $p-$dimensional, no permanece constante a medida que el sistema evoluciona se los denomina sistemas disipativos. Para los sistemas disipativos el volumen f\'asico tiende a contraerse a medida que pasa el tiempo, obteniendo para tiempos largos un comportamiento m\'as independiente de las condiciones iniciales. Esto quiere decir que, a medida que el sistema disipativo evoluciona en el tiempo, la trayectoria en el espacio de estados tiende a un punto, curva u objeto geom\'etrico, denominado atractor. Al conjunto de condiciones iniciales que dan lugar al atractor para $t \rightarrow \infty$ se le denomina \textit{Base de Atracci\'on}. Se pueden distinguir diferentes tipos de atractor seg\'un el objeto geom\'etrico del que se trate: punto fijo, ciclo l\'imite, superficies cerradas y atractores extraños.

Al igual que las condiciones iniciales pueden cambiar el comportamiento futuro del sistema, pequeños cambios en los par\'ametros pueden producir un cambio en las soluciones. El estudio de estos efectos se realiza mediante el an\'alisis del de la estabilidad estructural. Los autovalores que caracterizan la estabilidad de las soluciones, dependen de los par\'ametros del sistema. De este modo, un cambio en los par\'ametros pueden modificar la estabilidad de la solución. El cambio en el comportamiento de un sistema al modificar el valor de un par\'ametro se le llama bifurcaci\'on y al estudio de dichos efectos teor\'ia de bifurcaciones.

Se denomina punto fijo $X_{i0}$ a la solución para la cu\'al la trayectoria llega y se queda en ese valor fijo. Un punto fijo puede ser estable, si las trayectorias que empiezan cerca de \'el se mantienen siempre cerca de \'este, inestable o punto de silla. Para los espacios de estado $1-$dimensional el atractor es siempre un punto fijo.

Si la trayectoria en el espacio de estados es periódica rodeando a un punto fijo, a dicho punto fijo se le denomina CENTRO y a la curva cerrada CICLO L\'IMITE. Este ciclo l\'imite puede ser estable, si las trayectorias en la vecindad son atra\'idas, o inestables si son repelidas. Al nacimiento de un ciclo l\'imite desde un punto fijo se le conoce como bifurcaci\'on de Hopf. Otro tipo de bifurcación es la de doblamiento de periodo, en la que una solución periódica, de periodo $T$, pierde su estabilidad al cambiar un parámetro del sistema, siendo el nuevo atractor una solución periódica de periodo doble, $2T$.

La divergencia de trayectorias cercanas en el espacio de estados define el comportamiendo ca\'otico, para el cu\'al se han de cumplir tres condiciones:  Las trayectorias distintas no se pueden cortar, han de estar acotadas y las trayectorias cercanas han de diverger exponencialmente. Estas condiciones solo se cumplen para espacios de estados con dimensión $\geq 3$ y se denominan atractores extraños a los asociados a dichas trayectorias. La condici\'on de divergencia exponencial de dos trayectorias cercanas sobre el atractor viene definida por le ecuaci\'on \ref{eq:caos}.

	\begin{equation}
		d(t) = d_0 e^{\lambda t}
		\label{eq:caos}
	\end{equation}

En la ecuaci\'on \ref{eq:caos} el t\'ermino $d_0$ es la separaci\'on entre las trayectorias para $t = 0$ y $\lambda$ es el exponente de Lyapunov, que toma valores $\lambda > 0$ para el comportamiento ca\'otico. Otro indicador del comportamiento ca\'otico del sistema puede obtenerse a partir del espectro de potencia de la señal del sistema. El espectro de potencia $P(\omega)$ se define a partir de la transformada de Fourier de la señal $x(t)$ seg\'un la ecuaci\'on \ref{eq:DEP}.

	\begin{equation}
		P(\omega) = \lim_{T\rightarrow\infty} |\mathcal{F}(x) (\omega)|^2 = \lim_{T\rightarrow\infty} \left|\int_{0}^{T} x(t) e^{i\omega t} \mathrm{d}t \right| ^2
		\label{eq:DEP}
	\end{equation}

El espectro de potencias de un movimiento peri\'odico consiste en l\'ineas discretas bien definidas para ciertas frecuencias, mientras que para el caso del movimiento ca\'otico se obtiene una funci\'on ruidosa y de un gran ancho en frecuencias.
