El uso de dispositivos el\'ectricos de semiconductor supuso un avance enorme en la tecnolog\'ia, permitiendo desarrollar dispositivos m\'as eficientes y pequeños, conviertiendose en una parte fundamental de nuestra sociedad. Uno de estos dispositivos que han supuesto un gran avance son los l\'aseres de semiconductor, que ha descubierto un enorme campo de estudio, debido a sus importantes propiedades y a la multitud de aplicaciones en diferentes \'areas.

Este trabajo se centra en la creaci\'on de peines de frecuencia \'optica (OFC de sus siglas en ingl\'es) en un l\'aser de semiconductor de emisi\'on lateeral y modo discreto. Estos OFC presentan caracter\'isticas diversas en funci\'on de los par\'ametros que defienen su proceso de creaci\'on, debido a la din\'amica no lineal del sistema del l\'aser.

El objetivo de este trabajo es el estudio computacional del comportamiento de la diferentes regiones din\'amicas de los peines de frecuencia \'optica en l\'aseres de semiconductor con \gs\ e inyecci\'on \'optica. Este estudio permitir\'a entender los procesis f\'isicos relevantes en la formaci\'on de los OFC.

En este cap\'itulo se han introducido los conceptos te\'oricos relevantes de los temas tratados en este trabajo. Se ha realizado una breve introducci\'on sobre los l\'aseres de semiconductor, los procesos estoc\'asticos y la din\'amica no lineal, pasando a continuaci\'on a describir los peines de frecuencia \'optica y dos de los m\'etodos utilizados para su obtenci\'on: el \gs\ y la inyecci\'on \'optica.

	\addtocontents{toc}{\vspace{0.01cm}}
	\section{Láseres de Semiconductor}
		\label{Intr:LsrSmcdtr}
		
		\graphicspath{{./Introduccion/Figures/}}

La caracter\'istica principal de los semiconductores es la separaci\'on o gap entre la banda de valencia y la banda de conducci\'on, con un valor pequeño ($\varepsilon_g  = \varepsilon_C - \varepsilon_V \sim 0.1 - 3$ eV). Al tener un gap pequeño pueden darse saltos entre electrones de la banda de valencia y la banda de conducci\'on, apartir de la mediaci\'on de un fot\'on. Estas interacciones solo se pueden dar para energias del fot\'on $h\nu > \varepsilon_g$.

Las tres posibles formas de interacci\'on luz-materia son la emisi\'on espont\'anea, la emisi\'on estimulada y la absorci\'on. En un material semiconductor a $T = 0$K se tiene la banda de valencia completamente llena y la banda de conducci\'on vac\'ia. Al aumentar la temperatura se pueden dar saltos de los electrones a la banda de conducci\'on debido al pequeño gap. Al situarse un electr\'on en un nivel superior de energ\'ia, \'este tender\'a a volver a su estado de m\'inima energ\'ia recombinandose de nuevo en la banda de valencia, pudiendo emitir un fot\'on en el proceso. Este proceso es el denominado emisi\'on espont\'anea y es muy d\'ebil debido a que cada fot\'on puede tener diferente fase y direcci\'on.

	\begin{figure}[H]
		\centering
		\includegraphics[width=0.6\linewidth]{BV-BC.png}
		\caption{\label{Img:Saleh-BV-BC}Esquema de bandas con las tres interacciones luz-materia en un semiconductor: (a) absorci\'on, (b) emisi\'on espont\'anea y (c) emisi\'on estimulada \cite{saleh2019fundamentals}.}
	\end{figure}

Debido a la poca intensidad de la emisi\'on espont\'anea, el proceso clave para los l\'aseres de semiconductor, al igual que para el resto, es el dominio de la emisi\'on estimulada. Si sobre el material con un electr\'on excitado en la banda de conducci\'on, se incide con un fot\'on de frecuencia $\nu_0$ igual a la diferencia de energ\'ias entre el estado excitado del electr\'on y el fundamental, se forzar\'a al electr\'on a decaer al nivel de menor energ\'ia. El fot\'on emitido en la recombinaci\'on del electr\'on tendr\'a las mismas propiedades que el fot\'on incidente, puediendo estos dos fotones interactuar con m\'as electrones excitados y as\'i producir la amplificaci\'on de la radiaci\'on incidente. Sin embargo, para que esta amplificaci\'on continue, debe haber una cantidad suficiente de electrones excitados. Para ello se utilizan fuentes de bombeo para obtener la inversi\'on de poblaci\'on.

El medio en el cu\'al se producen la mayor parte de emisiones espont\'aneas se denomina medio activo y para los l\'aseres de semiconductor suelen ser heteroestructuras por capas conuna uni\'on $pn$ en la direcci\'on de avance de la corriente. Cuando la corriente inyectada al l\'aser es suficientemente grande, se obtienen suficientes electrones en la banda de conducci\'on para amplificar la luz. Una ventaja de los l\'aseres de semiconductor es su alta densidad de electrones, que permiten alcanzar grandes valores para la ganancia y unas distancias pequeñas de la cavidad, del orden de $\sim 1$ mm.

La direcci\'on del medio activo con respecto al haz de luz emitido permite diferenciar entre dos tipos de l\'aseres de semiconductor. Los l\'aseres de emisi\'on vertical (VCSEL) presentan el medio activo ortogonal al haz de luz mientras que para los de emisión lateral el medio activo es paralelo al haz de luz emitida. 

NO SE SI HE DE MENCIONAR ALGO SOBRE LASERES DE MODO DISCRETO


	\addtocontents{toc}{\vspace{0.01cm}}
	\section{Procesos Estocásticos}
		\label{Intr:PrcsEstcs}
		
		
	%En esta seccion se va a hablar de los procesos estocasticos del problema y de las ecuaciones diferenciales estocasticas, asi como de su resolucion.
	

Los procesos estoc\'asticos permiten describir la evoluci\'on temporal de un sistema con comportamiento aleatorio a partir de la probabilidad estad\'istica de \'este \cite{gardiner1985handbook}. El ejemplo m\'as relevante de los procesos estoc\'asticos es el movimiento Browniano, que describe la evoluci\'on de la posici\'on de una part\'icula en el seno de un fluido, con impactos frecuentes e irregulares (aleatorios) con las part\'iculas del fluido. El movimiento Browniano viene definido por la ecuaci\'on de Langevin y as\'i, se puede describir su evoluci\'on realizando el promedio con diferentes perturbaciones aleatorias. Para el caso de un experimento esto equivaldr\'ia a realizar el promedio de los resultados, repitiendo el experimento un n\'umero suficiente de veces, hasta obtener la distribuci\'on de probabilidad del proceso.  Una forma alternativa de resoluci\'on del movimiento Browniano es a partir de la ecuaci\'on de Fokker-Planck. Para la evolución unidimensional esta ecuación es la ecuacion de difusión \ref{eq:Fokker-Plank} que da la probabilidad de que la posición de la partícula, $\overline{\underline{X}}(t)$, esté en $x$ en un tiempo $t$ con $P = P(x, t)$, donde $D$ es el coeficiente de difusión.

	\begin{equation}
		\frac{\partial P}{\partial t} = D \frac{\partial^2 P}{\partial x^2}
		\label{eq:Fokker-Plank}
	\end{equation}

	Otro ejemplo de proceso estoc\'astico es el de luz emitida por un l\'aser, cuya principal fuente de aleatoriedad viene dada por la emisión espontánea. Cuando la distribuci\'on de probabilidad del sistema para un conjunto de variables, no varia para un desplazamiento en el tiempo, se dice que se trata de un proceso estoc\'astico estacionario y se cumple que:
	
	\begin{equation}
		\begin{matrix}
			P(x_1, t_1,..., x_n, t_n) = P(x_1, t_1+\delta t,..., x_n, t_n+\delta t) \\ \\
			P(x_1, t_1, x_2, t_2) = P(x_1, x_2, \tau) \rightarrow \textrm{      con    } \tau = t_2 - t_1
		\end{matrix}
	\end{equation}



As\'i, la autocorrelaci\'on $R(t_1, t_2) = E\left[\overline{\underline{X}}(t_1)\overline{\underline{X}}(t_2)\right] = R_\tau$ depende solo de la diferencia de tiempos $\tau = t_1 - t_2$. En la ecuación anterior $E\left[\overline{\underline{X}}(t_1)\overline{\underline{X}}(t_2)\right]$ es el valor medio del producto de $\overline{\underline{X}}(t_1)$ por $\overline{\underline{X}}(t_2)$ sobre distintos resultados del experimento. Para un proceso estacionario, para el que la media es contante respecto a $t$, se puede relacionar la autocorrelaci\'on y el espectro de potencia $S(\omega)$ mediante el teorema de Kintchine, y la transformada de Fourier.

	\begin{equation}
		\begin{matrix}
			S(\omega) = \int_{-\infty }^{\infty } R(\tau) e^{-i\omega\tau} \mathrm{d} \tau \\ \\
			S(\omega) = \lim_{T \rightarrow 0} E\left[\frac{1}{2T} \left| \int_{-T }^{T} \overline{\underline{X}}(t) e^{-i\omega t} \mathrm{d} t \right| ^2 \right ]
		\end{matrix}
		\label{eq:kintchine}
	\end{equation}

%En los procesos estoc\'asticos Markovianos las probabilidades condicionadas est\'an determinadas por el conocimiento del pasado m\'as reciente , pues $\tau_1 \geq \tau_2 \geq \textrm{...} \geq \tau_n$. Para que un proceso estoc\'astico sea Markoviano ha de cumplir las condiciones de \ref{eq:Markiviano}, pudiendo obtener la ecuaci\'on de Chapman Kolmog\'orov \ref{eq:Ch-Kgr}.

	%\begin{equation}
	%	\begin{matrix}
	%		P(x_1, t_1, x_2, t_2, ...| y_1, \tau_1, y_2, \tau_2, ...) = P(x_1, t_1, x_2, t_2, ...| y_1, \tau_1) \\ \\
	%		P(x_1, t_1, x_2, t_2, ..., x_n, t_n) = P(x_1, t_1|x_2, t_2)P(x_2, t_2|x_3, t_3)...P(x_n-1, t_n-1|x_n, t_n)P(x_n, t_n)
	%	\end{matrix}
	%	\label{eq:Markiviano}	
	%\end{equation}

	%\begin{equation}
	%	P(x_1, t_1|x_3, t_3) = \int \mathrm{d}x_2 P(x_1, t_1|x_2, t_2)P(x_2, t_2|x_3, t_3)
	%	\label{eq:Ch-Kgr}
	%\end{equation}

Hay un tipo especial de procesos estocásticos en los que las probabilidades condicionadas están determinadas por el conocimiento del pasado más reciente, los procesos Markovianos. Para ellos la densidad de probabilidad de que el proceso tome el valor $z$ en $t$, sabiendo que tomó el valor $y$ en $t'$ ($P(z, t| y, t')$), satisface la ecuaci\'on de Fokker Plank \ref{eq:FPE}.

	\begin{equation}
		\frac{\partial P(z, t|y, t'))}{\partial t} = \frac{\partial }{\partial z}\left(A(z, t)P(z, t|y, t') \right ) + \frac{1}{2} \frac{\partial^2 }{\partial z^2}\left(B(z, t)P(z, t|y, t') \right )	
		\label{eq:FPE}
	\end{equation}

Un desarrollo de la ecuaci\'on \ref{eq:FPE} con condición inicial $P(z, t|y, t) = \delta(z-y)$ permite hallar una densidad de probabilidad $P(z, t+\Delta t | y, t)$ gaussiana con media $y(t) + A(y, t)\Delta t$ y varianza $B\Delta t$.

	\begin{equation}
		Z(t+\Delta t) = y(t) + A(y, t)\Delta t + \eta(t) \sqrt{\Delta t}
	\end{equation}

El sistema evoluciona con un arrastre sistem\'atico $y(t) + A(y, t)\Delta t$ sobre el que se superpone la fluctuaci\'on $\eta(t)$, gaussiana de media cero y varianza $B$.

Las ecuaciones diferenciales estoc\'asticas vienen definidas por la ecuaci\'on de Langevin de la forma:

	\begin{equation}
		\frac{\mathrm{d} x}{\mathrm{d} t} = a(x, t) + b(x, t) \xi(t)
		\label{eq:SDE}
	\end{equation}

En la ecuaci\'on \ref{eq:SDE} aparece el t\'ermino aleatorio $\xi(t)$ que describe la fluctuaci\'on r\'apida e irregular, con $\left \langle \xi(t) \right \rangle \equiv 0$ y $\left \langle \xi(t)\xi(t') \right \rangle = \delta(t-t')$. A este t\'ermino se le conoce como ruido blanco, debido a que se obtiene que el espectro de potencia es $S(\omega) = 1$ seg\'un la ecuaci\'on \ref{eq:kintchine} para $R(\tau) = \left \langle \xi(t)\xi(t') \right \rangle$.

Se obtiene que las ecuaciones \ref{eq:FPE} y \ref{eq:SDE} son equivalentes, por lo que se pueden simular procesos estoc\'asticos mediante la resoluci\'on num\'erica de \ref{eq:FPE} o mediante la integraci\'on num\'erica de la ecuaci\'on diferencial estoc\'astica \ref{eq:SDE}. 

Para la integraci\'on num\'erica de la ecuaci\'on estoc\'astica \ref{eq:SDE} se ha utilizado el algoritmo de Euler, discretizando el valor de $t$ (d$t \approx \Delta t$) y que d$x \approx x(t+\Delta t) - x(t)$. Considerando estas aproximaciones y teniendo en cuenta su equivalencia con la ecuaci\'on \ref{eq:FPE} se ha obtenido la siguiente expresi\'on:

	\begin{equation}
		\begin{matrix}
		x(t+\Delta t) = x(t) + a(x, t)\Delta t + \eta(t)\sqrt{\Delta t} \\ \\
		\eta(t) \textrm{ Gaussiano} \rightarrow \eta = \sqrt{V[\eta]} Z + E[\eta] = bZ
		\end{matrix}
		\label{eq:Gauss}
	\end{equation}

En la ecuaci\'on \ref{eq:Gauss} se muestran los t\'erminos del ruido gaussiano, siendo $E[\eta] = 0$ su media, $V[\eta] = b^2$ la varianza y $Z_i = N(0, 1)$ una distribuci\'on de probabilidad gaussiana. De esta forma, la soluci\'on a la ecuaci\'on \ref{eq:SDE}, considerando $x_i = x(t_i)$, queda \cite{gardiner1985handbook}:

	\begin{equation}
		x_{i+1} = x_i + a(x_i, t_i)\Delta t + b(x_i, t_i) Z_i\sqrt{\Delta t}
	\end{equation}


	\addtocontents{toc}{\vspace{0.01cm}}
	\section{Dinámica No Lineal}
		\label{Intr:NonLnr}
		
		
	%En esta seccion se hablara de la dinamica no lineal, bifurcaciones, M.T., y de las dos formas de las que se puede obtener dinamica no lineal en nuestro laser de semiconductor

Los sistemas din\'amicos son aquellos que pueden ser descritos mediante ecuaciones diferenciales ordinarias $\frac{\mathrm{d}\vec{x}}{\mathrm{d}t} = \vec{f}(\vec{x})$ con $\vec{x}$ continua en el espacio. Estos sistemas son a su vez sistemas deterministas, pues es posible, en principio, predecir su comportamiento futuro a partir del conocimiento preciso de su estado en un tiempo $t$ anterior. Las variables $\vec{x} = (x_1,..., x_p)$ forman el espacio de estados, que para un sistema mec\'anico es el espacio de fases con dimensi\'on $p = 2n$, siendo $n$ el n\'umero de grados de libertad.

El estudio de los sistemas din\'amicos cuyas ecuaciones no evolucionan con un comportamiento lineal se denomina din\'amica no lineal. Estos sistemas pueden realizar un comportamiento casi aleatorio y err\'atico, obteniendo caos determinista. Los sistemas ca\'oticos tambi\'en se caracterizan por su gran sensibilidad a las condiciones iniciales, o \textit{divergencia de trayectorias cercanas}. Esta divergencia puede producir diferencias muy notables al cabo de un cierto tiempo en las trayectorias de dos sistemas con condiciones iniciales muy similares, obteniendo un comportamiento impredecible del sistema. Sin embargo, los sistemas ca\'oticos se consideran deterministas en el sentido de que es posible predecir su comportamiento futuro conociendo las condiciones iniciales exactas, a partir de la integraci\'on de las ecuaciones diferenciales del sistema.

Sin embargo, en los experimentos f\'isicos reales es casi imposible recrear las condiciones iniciales con suficiente exactitud. Es por ello que cobra mayor importancia el concepto de estabilidad din\'amica, definiendo \'esta como la sensibilidad de una soluci\'on, a pequeñas variaciones de las condiciones iniciales. Si se tienen dos trayectorias $x(t)$ e $y(t)$ para las cuales en un instante dado $t_0$, $x(t_0$ e $y(t_0)$ est\'an suficientemente cerca, $x(t)$ ser\'a uniformemente estable en el sentido de \textit{Lyapunov} si $x(t)$ e $y(t)$ se mantienen pr\'oximos en todo el tiempo.

Los sistemas en los cuales el volumen f\'asico, o volumen en el espacio f\'asico $p-$dimensional, no permanece constante a medida que el sistema evoluciona se los denomina sistemas disipativos. Para los sistemas disipativos el valomuen f\'asico tiende a contraerse a medida que pasa el tiempo, obteniendo para tiempos largos un comportamiento m\'as independiente de las condiciones iniciales. Esto quiere decir que, a medida que el sistema disipativo evoluciona en el tiempo, la trayectoria en el espacio de estados tiende a un punto, curva u objeto geom\'etrico, denominado atractor. Al conjunto de condiciones iniciales que dan lugar al atractor para $t \rightarrow \infty$ se le denomina \textit{Base de Atracci\'on}. Se pueden distingir diferentes tipos de atractor seg\'un el objeto geom\'etrico del que se trate: punto fijo, ciclo l\'imite, superficies cerradas y atractores extraños. Si para un conjunto dado de par\'ametros de un sistema se obtienen m\'as de un atractor, se llama separatriz a las condiciones iniciales que se encuentran en la frontera entre dos o m\'as bases de atracci\'on.

Al igual que las condiciones iniciales pueden cambiar de comportamiento futuro del sistema, pequeños cambios en los par\'ametros pueden producir un cambio en las soluciones. El estudio de estos efectos se realiza mediante el an\'alisis del de la estabilidad estructural. Los autovalores de los atractores dependen de los par\'ametros del sistema, y as\'i un cambio en los par\'ametros pueden modificar el tipo de atractor. El cambio en el comportamiento de un sistema al modificar el valor de un par\'ametro se le llama bifurcaci\'on y al estudio de dichos efectos teor\'ia de bifurcaciones.

Se denomina punto fijo $X_i0$ (con $i$ perteneciende a la base de atracci\'on) al atractor para el cu\'al la trayectoria llega y se queda a un valor fijo. Un punto fijo puede ser estable, si las trayectorias que empiezan cerca de \'el se mantienen siempre cerca de \'este, inestables o de punto de silla. Para los espacios de estado $1-$dimensional el atractor es siempre un punto fijo.

Si la trayectoria en el espacio de estados es periodica respecto a un punto fijo, a dicho punto fijo se le denomina CENTRO y a la curva cerrada CICLO L\'IMITE. Este ciclo l\'imite puede ser estable, si las trayectorias en la vecindad son atra\'idas, o inestables si son repelidas. Al nacimiento de un ciclo l\'imite desde otro atractor se le conoce como bifurcaci\'on de Hopf.

La divergencia de trayectorias cercanas en el espacio de estados define el comportamiendo ca\'otico, para el cu\'al se han de cumplit tres condiciones:  Las trayectorias distintas no se pueden cortar, han de estar acotadas y las trayectorias cercanas han de diverger exponencialmente. Estas condiciones solo se cumplen para espacios de estados con dimensión $\geq 3$ y se denominan atractores extraños a los asociados a dichas trayectorias. La condici\'on de divergencia exponencial de dos trayectorias cercanas sobre el atractor viene definida por le ecuaci\'on \ref{eq:caos}.

	\begin{equation}
		d(t) = d_0 e^{\lambda t}
		\label{eq:caos}
	\end{equation}

En la ecuaci\'on \ref{eq:caos} el t\'ermino $d_0$ es la separaci\'on entre las trayectorias para $t = 0$ y $\lambda$ es el esponente de Lyapunov, que toma valores $\lambda > 0$ para el comportamiento ca\'otico. Otro indicador del comportamiento ca\'otico del sistema puede obtenerse a partir del espectro de potencia de la señal del sistema. El espectro de potencia $P(\omega)$ se define a partir de la transformada de Fourier de la señal $x(t)$ seg\'un la ecuaci\'on \ref{eq:DEP}.

	\begin{equation}
		P(\omega) = |\mathcal{F}(x(t)) (\omega)|^2 = \left|\lim_{T\rightarrow\infty} \int_{0}^{T} x(t) e^{i\omega t} \mathrm{d}t \right| ^2
		\label{eq:DEP}
	\end{equation}

El espectro de potencias de un movimineto peri\'odico consiste en l\'ineas discretas bien definidas para ciertas frecuencias, mientras que para el caso del movimiento ca\'otico se obtiene una funci\'on ruidosa para un mayor rango de frecuencias.


	\addtocontents{toc}{\vspace{0.01cm}}
	\section{Peines de Frecuencia Óptica}
		\label{Intr:OFC}
		
		

Las l\'ineas de emisi\'on de un l\'aser de semiconductor se obtienen del espectro \'optico, definido como el m\'odulo cuadrado de la transformada de Fourier del campo el\'ectrico, que depende de la potencia del l\'aser y de la fase \'optica de \'este. Para un l\'aser en corriente cont\'inua, el campo el\'ectrico, sin ningún tipo de ruido,  alcanza un valor constante y as\'i, su transformada de Fourier es una funci\'on delta de Dirac $\delta(\nu-\nu_0)$, obteniendo en el espectro una \'unica l\'inea de emisi\'on bien definida para una cierta frecuencia $\nu_0$.

	\begin{figure}[H]
		\centering
		\includegraphics[width=1.0\linewidth]{OFC.png}
		\caption{\label{Img:FFtPulsos}Esquemas de la relaci\'on entre la funci\'on de la evolución temporal de una señal (fila superior) y su transformada de Foutier en el espacio de frecuencias $\nu$ (fila inferior), para tres tipos de señales: (a) tren de pulsos infinitamente estrechos, (b) un \'unico pulso con anchura (FWHM) $\delta t$ y (c) un tren de pulsos con anchura (FWHM) $\delta t$.}
	\end{figure}

Si se tiene una señal formada por un tren de pulsos infinitamente estrechos separados un tiempo $T$, se sabe por su transformada de Fourier que el espectro \'optico en frecuencias ha de ser otro tren de pulsos infinitamente estrechos, pero separados $1/T$ (Figura \ref{Img:FFtPulsos} (a)). Sin embargo, en los experimentos no es posible obtener pulsos infinitamente estrechos sino que se tiene pulsos con una cierta anchura FWHM (siglas en ingl\'es de anchura a media altura). Su espectro \'optico ser\'a otro pulso con FWHM $\delta\nu$ (Figura \ref{Img:FFtPulsos} (b)), donde se tiene que se ha de cumplir que el producto $\delta\nu\delta t$ es contante. En la ecuaci\'on \ref{eq:types} se muestran diferentes valores del producto $\delta\nu\delta t$ seg\'un la forma del pulso en $t$.

	\begin{equation}
		\delta\nu\delta t = 
			\left\{\begin{matrix}
				& 0.315 & \textrm{sech}^2 \\
				& 0.44 & \textrm{Gaussiana}
			\end{matrix}\right.
		\label{eq:types}
	\end{equation}

	Un tren de pulsos con ancho $\delta t$ ser\'a la combinaci\'on de las dos señales anteriores y su expresi\'on vendr\'a dada por la convoluci\'on de ambas funciones. Su transformada de Fourier es el producto de las transformadas de Fourier de las funciones anteriores, obteniendo un espectro formado por pulsos estrechos con una separaci\'on entre ellos de $1/T$ y cuya envolvente viene dada por la forma del espectro del pulso individual (Figura \ref{Img:FFtPulsos} (c)). A este tipo de espectros se les denomina peines.

Los peines de frecuencia \'optica (OFC por sus siglas en ingl\'es) son fuentes \'opticas formados por un gran n\'umero l\'ineas de emisi\'on con un espaciado preciso y equidistante. Los OFC vienen caracterizados tanto por la forma de la envolvente como por la separaci\'on entre los picos, obteniendo OFC de mayor calidad para envolventes anchas y regulares. Esto se obtiene para campos el\'ectricos con pulsos regulares, estrechos e intensos.

Cabe recordar que los espectros \'opticos se obtienen a partir del campo el\'ectrico, cuyo comportamiento puede ser descrito mediante una ecuaci\'on de ondas de la que obtenemos una fase \'optica $\Phi$ y una amplitud relacionada con la potencia del l\'aser. Mientras que para la potencia puede ser descrita con el desarrollo anterior, los efectos de la fase \'optica son diferentes. De esta forma, los efectos que se observan al estudiar los OFC son el resultado de la evolución de amplitud y fase óptica. Uno de los efectos de la fase \'optica en los OFC se puede observar en la anchura de las l\'ineas espectrales, que disminuye para fases menos aleatorias.

Pese a que existen diferentes mecanismos de creaci\'on de OFC, en este trabajo se estudiar\'an solo los m\'etodos de generaci\'on de OFC mediante \gs\ y por inyecci\'on \'optica.

	\subsection{Encendido por Ganancia}
		\label{Intr:OFC:GS}

		El \gs\ (\textit{Gain-Switching} en ingl\'es) es una t\'ecnica mediante la cu\'al se alcanza r\'apidamente un alto valor para la ganancia del l\'aser \cite{principles}. Esta t\'ecnica permite generar pulsos del l\'aser de corta duraci\'on y grandes picos de potencia, pudiendo obtener OFC de gran calidad. El \gs\ consiste en conseguir que la inversi\'on de poblaci\'on, y por tanto la ganancia, alcance un valor muy por encima del valor umbral antes de que la densidad de fotones tenga tiempo de alcanzar un nivel suficiente para reducir la inversi\'on. 
		
		El uso de pulsos de bombeo suficientemente r\'apidos, corriente eléctrica en el caso de láseres de semiconductor, permite alcanzar la condici\'on de la inversi\'on de poblaci\'on, alcanzando el \gs. Mediante el uso de corriente de inyecci\'on modulada por una funci\'on sinusoidal se puede controlar la forma del pulso de bombeo a partir de la amplitud y frecuencia, controlando los valores \'optimos de ancho y potencia de pulsos ópticos para el \gs.

	\subsection{Inyección Óptica}

		Otro m\'etodo de generaci\'on de OFC es mediante la inyecci\'on \'optica. \'Esta consiste en inyectar fotones provenientes de un segundo l\'aser al l\'aser de semiconductor. Bajo determinadas condiciones de potencia y frecuencia del láser que inyecta, se pueden obtener OFC. Puesto que se va a trabajar con l\'aseres de semiconductor, cabe destacar la gran sensibilidad de \'estos a la inyecci\'on \'optica debida entre otras cosas al acoplamiento amplitud-fase a trav\'es del factor de ensanchamiento del ancho de l\'inea \cite{tfgPopp}.

		Para un l\'aser sin inyecci\'on de luz, la fase evoluciona de manera aleatoria en el tiempo, principalmente debido a la emisión espontánea, y as\'i se obtienen anchos de l\'inea grandes (del orden del MHz para un láser monomodo). Sin embargo, al realizarse la inyecci\'on de luz las caracter\'isticas de la fase del l\'aser inyectado pasan a estar determinadas por la inyecci\'on , pudiendo obtener un fase menos aleatoria si el espectro óptico del láser que inyecta es más estrecho. De esta forma se tiene un menor ruido en la fase, obteniendo picos m\'as estrechos en el OFC.

		Este m\'etodo tambi\'en puede producir otro fen\'omeno bajo unas condiciones determinadas, conocido como bloqueo por inyecci\'on. Si se tiene una inyecci\'on \'optica de fotones de frecuencia diferente a la del l\'aser inyectado, bajo ciertas condiciones se puede dar que el l\'aser inyectado comience a emitir en la frecuencia del l\'aser que inyecta, desapareciendo la emisi\'on del l\'aser inyectado a la frecuencia de emisión en solitario.

	\subsection{Aplicaciones}

		Tal y como se ha descrito anteriormente, los OFC presentan l\'ineas de emisi\'on bien definidas, perfectamente equiespaciadas y con una fuerte correlaci\'on en la fase \cite{desi2017development}. \'Esto les convierte en una herramienta de gran inter\'es para la espectroscop\'ia y las cominucaciones \'opticas.

		Los OFC permiten obtener varias l\'ineas de emisi\'on bien definidas y equiespaciadas para diferentes longitudes de onda a partir de la emisi\'on de un \'unico l\'aser. De esta forma permite sustituir sistemas formados por multiples l\'aseres independientes, disminuyendo costes, consumo de potencia y complejidad. Estos sistemas con láseres independientes se utilizan en la actualidad en comunicaciones ópticas de alta velocidad y se les llama DWDM (\textit{Dnse wavelength division multiplexing} en inglés). Si a esto se le añade la capacidad de ajustar la frecuencia de separación entre las líneas o la longitud de onda de éstas, se obtiene unas cualidades de gran importancia para su uso en comunicaciones \'opticas. Esta capacidad de ajuste se puede obtener con los OFC en láseres de semiconductor en \gs. Además, la correlación de fases del OFC presenta multiples ventajas para la transmisión por fibra óptica, pudiendo ser utilizada para cancelar los efectos no lineales que distorsionan la información que viaja por la fibra, pudiendo recuperar dicha información \cite{temprana2015overcoming}. Para las aplicaciones en comunicaciones ópticas de alta velocidad, son deseables OFC con una gran separación entre líneas. Sin embargo, para las aplicaciones de espectroscopía óptica se requieren OFC con separación entre líneas pequeña, para obtener espectros con buena resolución.


	%\addtocontents{toc}{\vspace{0.01cm}}
	%\section{Objetivo del Estudio}
	%	\label{Intr:Obj}
		
		%\input{}
